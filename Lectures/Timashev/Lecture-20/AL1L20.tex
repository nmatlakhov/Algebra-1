\documentclass[12pt]{article}
\usepackage[left=1cm, right=1cm, top=2cm,bottom=1.5cm]{geometry} 

\usepackage[parfill]{parskip}
\usepackage[utf8]{inputenc}
\usepackage[T2A]{fontenc}
\usepackage[russian]{babel}
\usepackage{enumitem}
\usepackage[normalem]{ulem}
\usepackage{amsfonts, amsmath, amsthm, amssymb, mathtools,xcolor}
\usepackage{blkarray}

\usepackage{tabularx}
\usepackage{hhline}

\usepackage{accents}
\usepackage{fancyhdr}
\pagestyle{fancy}
\renewcommand{\headrulewidth}{1.5pt}
\renewcommand{\footrulewidth}{1pt}

\usepackage{graphicx}
\usepackage[figurename=Рис.]{caption}
\usepackage{subcaption}
\usepackage{float}

%%Наименование папки откуда забирать изображения
\graphicspath{ {./images/} }

%%Изменение формата для ввода доказательства
\renewcommand{\proofname}{$\square$  \nopunct}
\renewcommand\qedsymbol{$\blacksquare$}

%%Изменение отступа на таблицах
\addto\captionsrussian{%
	\renewcommand{\proofname}{$\square$ \nopunct}%
}
%% Римские цифры
\newcommand{\RN}[1]{%
	\textup{\uppercase\expandafter{\romannumeral#1}}%
}

%% Для удобства записи
\newcommand{\MR}{\mathbb{R}}
\newcommand{\MC}{\mathbb{C}}
\newcommand{\MQ}{\mathbb{Q}}
\newcommand{\MN}{\mathbb{N}}
\newcommand{\MZ}{\mathbb{Z}}
\newcommand{\MTB}{\mathbb{T}}
\newcommand{\MTI}{\mathbb{I}}
\newcommand{\MI}{\mathrm{I}}
\newcommand{\MCI}{\mathcal{I}}
\newcommand{\MJ}{\mathrm{J}}
\newcommand{\MH}{\mathrm{H}}
\newcommand{\MT}{\mathrm{T}}
\newcommand{\MU}{\mathcal{U}}
\newcommand{\MV}{\mathcal{V}}
\newcommand{\MB}{\mathcal{B}}
\newcommand{\MF}{\mathcal{F}}
\newcommand{\MW}{\mathcal{W}}
\newcommand{\ML}{\mathcal{L}}
\newcommand{\MP}{\mathcal{P}}
\newcommand{\VN}{\varnothing}
\newcommand{\VE}{\varepsilon}
\newcommand{\dx}{\, dx}
\newcommand{\dy}{\, dy}
\newcommand{\dz}{\, dz}
\newcommand{\dd}{\, d}


\theoremstyle{definition}
\newtheorem{defn}{Опр:}
\newtheorem{rem}{Rm:}
\newtheorem{prop}{Утв.}
\newtheorem{exrc}{Упр.}
\newtheorem{problem}{Задача}
\newtheorem{lemma}{Лемма}
\newtheorem{theorem}{Теорема}
\newtheorem{corollary}{Следствие}

\newenvironment{cusdefn}[1]
{\renewcommand\thedefn{#1}\defn}
{\enddefn}

\DeclareRobustCommand{\divby}{%
	\mathrel{\text{\vbox{\baselineskip.65ex\lineskiplimit0pt\hbox{.}\hbox{.}\hbox{.}}}}%
}
\DeclareRobustCommand{\ndivby}{\mkern-1mu\not\mathrel{\mkern4.5mu\divby}\mkern1mu}


%Короткий минус
\DeclareMathSymbol{\SMN}{\mathbin}{AMSa}{"39}
%Длинная шапка
\newcommand{\overbar}[1]{\mkern 1.5mu\overline{\mkern-1.5mu#1\mkern-1.5mu}\mkern 1.5mu}
%Функция знака
\DeclareMathOperator{\sgn}{sgn}

%Функция ранга
\DeclareMathOperator{\rk}{\text{rk}}
\DeclareMathOperator{\diam}{\text{diam}}


%Обозначение константы
\DeclareMathOperator{\const}{\text{const}}

\DeclareMathOperator{\codim}{\text{codim}}

\DeclareMathOperator*{\dsum}{\displaystyle\sum}
\newcommand{\ddsum}[2]{\displaystyle\sum\limits_{#1}^{#2}}

%Интеграл в большом формате
\DeclareMathOperator{\dint}{\displaystyle\int}
\newcommand{\ddint}[2]{\displaystyle\int\limits_{#1}^{#2}}
\newcommand{\ssum}[1]{\displaystyle \sum\limits_{n=1}^{\infty}{#1}_n}

\newcommand{\smallerrel}[1]{\mathrel{\mathpalette\smallerrelaux{#1}}}
\newcommand{\smallerrelaux}[2]{\raisebox{.1ex}{\scalebox{.75}{$#1#2$}}}

\newcommand{\smallin}{\smallerrel{\in}}
\newcommand{\smallnotin}{\smallerrel{\notin}}

\newcommand*{\medcap}{\mathbin{\scalebox{1.25}{\ensuremath{\cap}}}}%
\newcommand*{\medcup}{\mathbin{\scalebox{1.25}{\ensuremath{\cup}}}}%

\makeatletter
\newcommand{\vast}{\bBigg@{3.5}}
\newcommand{\Vast}{\bBigg@{5}}
\makeatother

%Промежуточное значение для sup\inf, поскольку они имеют разную высоту
\newcommand{\newsup}{\mathop{\smash{\mathrm{sup}}}}
\newcommand{\newinf}{\mathop{\mathrm{inf}\vphantom{\mathrm{sup}}}}

%Скалярное произведение
\newcommand{\inner}[2]{\left\langle #1, #2 \right\rangle }
\newcommand{\linsp}[1]{\left\langle #1 \right\rangle }
\newcommand{\linmer}[2]{\left\langle #1 \vert #2\right\rangle }

%Подпись символов снизу
\newcommand{\ubar}[1]{\underaccent{\bar}{#1}}

%% Шапка для букв сверху
\newcommand{\wte}[1]{\widetilde{#1}}
\newcommand{\wht}[1]{\widehat{#1}}
\newcommand{\ovl}[1]{\overline{#1}}

%%Трансформация Фурье
\newcommand{\fourt}[1]{\mathcal{F}\left(#1\right)}
\newcommand{\ifourt}[1]{\mathcal{F}^{-1}\left(#1\right)}

%%Символ вектора
\newcommand{\vecm}[1]{\overrightarrow{#1\,}}

%%Пространстов матриц
\newcommand{\matsq}[1]{\operatorname{Mat}_{#1}}
\newcommand{\mat}[2]{\operatorname{Mat}_{#1, #2}}

%Оператор для действ и мнимых чисел
\DeclareMathOperator{\IM}{\operatorname{Im}}
\DeclareMathOperator{\RE}{\operatorname{Re}}
\DeclareMathOperator{\li}{\operatorname{li}}
\DeclareMathOperator{\GL}{\operatorname{GL}}
\DeclareMathOperator{\SL}{\operatorname{SL}}
\DeclareMathOperator{\Char}{\operatorname{char}}
\DeclareMathOperator\Arg{Arg}

%Делимость чисел
\newcommand{\modn}[3]{#1 \equiv #2 \; (\bmod \; #3)}


%%Взятие в скобки, модули и норму
\newcommand{\parfit}[1]{\left( #1 \right)}
\newcommand{\modfit}[1]{\left| #1 \right|}
\newcommand{\sqparfit}[1]{\left\{ #1 \right\}}
\newcommand{\normfit}[1]{\left\| #1 \right\|}

%%Функция для обозначения равномерной сходимости по множеству
\newcommand{\uconv}[1]{\overset{#1}{\rightrightarrows}}
\newcommand{\uconvm}[2]{\overset{#1}{\underset{#2}{\rightrightarrows}}}


%%Функция для обозначения нижнего и верхнего интегралов
\def\upint{\mathchoice%
	{\mkern13mu\overline{\vphantom{\intop}\mkern7mu}\mkern-20mu}%
	{\mkern7mu\overline{\vphantom{\intop}\mkern7mu}\mkern-14mu}%
	{\mkern7mu\overline{\vphantom{\intop}\mkern7mu}\mkern-14mu}%
	{\mkern7mu\overline{\vphantom{\intop}\mkern7mu}\mkern-14mu}%
	\int}
\def\lowint{\mkern3mu\underline{\vphantom{\intop}\mkern7mu}\mkern-10mu\int}

%%След матрицы
\DeclareMathOperator*{\tr}{tr}

\makeatletter
\renewcommand*\env@matrix[1][*\c@MaxMatrixCols c]{%
	\hskip -\arraycolsep
	\let\@ifnextchar\new@ifnextchar
	\array{#1}}
\makeatother


%% Переопределение функции хи, чтобы выглядела более приятно
\makeatletter
\@ifdefinable\@latex@chi{\let\@latex@chi\chi}
\renewcommand*\chi{{\@latex@chi\smash[t]{\mathstrut}}} % want only bottom half of \mathstrut
\makeatletter

\setcounter{MaxMatrixCols}{20}

\begin{document}
\lhead{Алгебра-\RN{1}}
\chead{Тимашев Д.А.}
\rhead{Лекция - 20}
\section*{Дроби}
Пусть $A$ - произвольная область целостности.
\begin{defn}
	В множестве $A \times \left(A\setminus \{0\}\right)$ будем говорить, что $(a,b) \sim (a',b')$, если $a{\cdot}b' = a'{\cdot}b$.
\end{defn}
\begin{prop}
	Заданное отношение $\sim$ является отношением эквивалентности.
\end{prop}

\begin{defn}
	Классы эквивалентности отношения $\sim$ назовём \uwave{дробями} элементов из кольца $A$.
\end{defn}

\textbf{\uline{Обозначение}}: $\dfrac{a}{b}$ - класс, содержащий $(a,b)$.

\textbf{\uline{Правило пропорции}}: $\dfrac{a}{b} = \dfrac{a'}{b'} \Leftrightarrow a{\cdot}b' = a'{\cdot}b$. В частности, из правила следует: $\dfrac{a}{b} = \dfrac{ac}{bc}, \, \forall c \neq 0$.

\textbf{\uline{Операции над дробями}}:
\begin{enumerate}
	\item[($+$):] $\forall \, \dfrac{a}{b}, \dfrac{c}{d} \in A \times \left(A\setminus \{0\}\right), \, \dfrac{a}{b} + \dfrac{c}{d} = \dfrac{ad + bc}{bd}$;
	\item[($\, \cdot\, $):] $\forall \,\dfrac{a}{b}, \dfrac{c}{d} \in A \times \left(A\setminus \{0\}\right), \, \dfrac{a}{b}{\cdot}\dfrac{c}{d} = \dfrac{a{\cdot}c}{b{\cdot}d}$;
\end{enumerate}
\textbf{\uline{Корректность}}: Пусть $\dfrac{a}{b} = \dfrac{a'}{b'}, \, ab' = a'b$, тогда: 
\begin{enumerate}
	\item[($+$):] $\dfrac{ad + bc}{bd} = \dfrac{ab'd + bb'c}{bb'd} = \dfrac{a'bd + bb'c}{bb'd} = \dfrac{a'd + b'c}{b'd} \Rightarrow$ операция сложения - корректна;
	\item[($\, \cdot\, $):] $\dfrac{ac}{bd} = \dfrac{ab'c}{bb'd} = \dfrac{a'c}{b'd} \Rightarrow$ операция умножения - корректна;
\end{enumerate}

\begin{defn}
	Множество всех дробей (классов эквивалентностей) $Q(A)$ с введенными операциями сложения и умножения дробей называется \uwave{полем дробей} или \uwave{полем частных} или \uwave{полем отношений} кольца $A$:
	$$
		Q(A) = \left\{\dfrac{a}{b}\mid a,b \in A, \, b \neq 0\right\}
	$$
\end{defn}
\textbf{\uline{Проверка аксиом поля}}:
\begin{enumerate}[label=\arabic*)]
	\item[1)-5)] \textbf{Коммутативность, ассоциативность, дистрибутивность сложения и умножения}: \\
	Приводим дроби к общему знаменателю $\Rightarrow$ тождества сводятся к соответствующим тождествам для числителей, которые выполнены, поскольку это элементы коммутативного, ассоциативного кольца $A$. Например, коммутативность сложения:
	$$
		\dfrac{a}{b} + \dfrac{c}{b} = \dfrac{ab + bc}{b^2} =\dfrac{ab + cb}{b^2} =\dfrac{(a + c)b}{bb} = \dfrac{a + c}{b} = \dfrac{c + a}{b} = \dfrac{c}{b} + \dfrac{a}{b}
	$$
	где во четвертом равенстве мы воспользовались правилом пропорции. Аналогично, для коммутативности умножения:
	$$
		\dfrac{a}{b}{\cdot}\dfrac{c}{b} = \dfrac{ac}{b^2}  = \dfrac{ca}{b^2} = \dfrac{c}{b}{\cdot}\dfrac{a}{b}
	$$
	\setcounter{enumi}{5}
	\item \textbf{Существование нулевого элемента}: $\forall \, \dfrac{a}{b} \in Q(A)$:
	$$
		\exists\, \dfrac{0}{1} \in Q(A) \colon  \dfrac{a}{b} + \dfrac{0}{1} = \dfrac{a{\cdot}1 + b{\cdot}0}{b{\cdot}1} = \dfrac{a}{b}
	$$
	Также заметим, что $\dfrac{0}{1} = \dfrac{0}{b}, \, \forall b \neq 0$;
	\item \textbf{Существование единичного элемента}: $\forall \, \dfrac{a}{b} \in Q(A)$:
	$$
		\exists\, \dfrac{1}{1} \in Q(A) \colon \dfrac{a}{b}{\cdot}\dfrac{1}{1} = \dfrac{a{\cdot}1}{b{\cdot}1} = \dfrac{a}{b}
	$$
	\item \textbf{Существование противоположного элемента}: $\forall \, \dfrac{a}{b} \in Q(A)$:
	$$
		\exists\, -\left(\dfrac{a}{b}\right) = \dfrac{-a}{b} \in Q(A) \colon \dfrac{a}{b} + \dfrac{-a}{b} = \dfrac{ab + b(-a)}{bb} = \dfrac{(a + (-a))b}{bb} = \dfrac{a + (-a)}{b} = \dfrac{0}{b} = \dfrac{0}{1}
	$$
	\item \textbf{Существование обратного элемента}: $\forall \, \dfrac{a}{b} \in Q(A), \, \dfrac{a}{b} \neq \dfrac{0}{1} \Rightarrow a{\cdot}1 = a \neq 0{\cdot}b  = 0$:
	$$
		\exists\, \left(\dfrac{a}{b}\right)^{-1} = \dfrac{b}{a} \in Q(A) \colon \dfrac{a}{b} {\cdot}\dfrac{b}{a} = \dfrac{ab}{ba} = \dfrac{ab}{ab} = \dfrac{1}{1}
	$$
\end{enumerate}
Это поле хорошо тем, что оно содержит в себе исходное кольцо $A$, то есть мы расширили $A$ до поля.
\begin{prop}
	Поле дробей $Q(A)$ содержит подкольцо: $\dfrac{A}{1} = \left\{\dfrac{a}{1} \mid a \in A\right\} \simeq A$.
	При отождествлении $\dfrac{A}{1}$ с $A$ верно следующее:
	$$
		\forall q \in Q(A), \, \exists, \, a,b\in A, \, b\neq 0 \colon q = a{\cdot}b^{-1}
	$$
\end{prop}
\begin{proof}
	Проверим замкнутость множества $\dfrac{A}{1}$ относительно сложения и умножения:
	$$
		\forall \, \dfrac{a}{1}, \dfrac{b}{1} \in \dfrac{A}{1}, \, \dfrac{a}{1} \pm \dfrac{b}{1} = \dfrac{a \pm b}{1} \in \dfrac{A}{1}
	$$
	$$
		\forall \, \dfrac{a}{1}, \dfrac{b}{1} \in \dfrac{A}{1}, \, \dfrac{a}{1}{\cdot}\dfrac{b}{1} = \dfrac{ab}{1} \in \dfrac{A}{1}
	$$
	Следовательно, $\dfrac{A}{1}$ является подкольцом. Соответствие $\varphi \colon A \to \dfrac{A}{1}, \, \varphi(a) = \dfrac{a}{1}$ является изоморфизмом:
	\begin{enumerate}[label=\arabic*)]
		\item \textbf{Согласованность с операциями}:
		$$
			\forall a,b \in A, \, \varphi(a \pm b) = \dfrac{a \pm b}{1} = \dfrac{a}{1} \pm \dfrac{b}{1} = \varphi(a) \pm \varphi(b) 
		$$
		$$
			\forall a,b \in A,\, \varphi(a{\cdot}b) = \dfrac{ab}{1} = \dfrac{a}{1}{\cdot}\dfrac{b}{1} = \varphi(a){\cdot}\varphi(b) 
		$$
		\item \textbf{Биективность}: $\dfrac{a}{1} = \dfrac{a'}{1} \Leftrightarrow a = a{\cdot}1 = a'{\cdot}1 = a' \Rightarrow$ будет взаимнооднозначное соответствие;
	\end{enumerate}

	Пусть $q \in Q(A)$, тогда: 
	$$
		q = \dfrac{a}{b} = \dfrac{a}{1}{\cdot}\dfrac{1}{b} = \dfrac{a}{1}{\cdot}\left(\dfrac{b}{1}\right)^{-1} = a b^{-1}
	$$
\end{proof}
\begin{rem}
	Полученное расширение кольца до поля в каком-то смысле является минимальным, поскольку мы присоеденили то, что нужно присоединить, чтобы получить поле.
\end{rem}
\textbf{Итог}: Поскольку мы доказали, что дробь вида $\dfrac{a}{b}$ является частным дробей, отождествляемых с элементами исходного кольца $ab^{-1}$, то поэтому знак дроби можно понимать как знак деления.

\textbf{Примеры}:
\begin{enumerate}[label=\arabic*)]
	\item $Q(\MZ) = \MQ$;
	\item $Q(K[x]) = K(x)$ - \uwave{поле рациональных дробей} от одной переменной с коэффициентами из поля $K$;
\end{enumerate}

\section*{Поле рациональных дробей}
\begin{defn}
	\uwave{Полем рациональных дробей} от одной переменной с коэффициентами из поля $K$ называется поле дробей от кольца многочленов $K[x]$: $Q(K[x]) = K(x)$.
\end{defn}

\begin{defn}
	Рациональная дробь: $r = \dfrac{f}{g} \in K(x)$ задаёт \uwave{рациональную функцию}: 
	$$
		r\colon K \setminus \{x_1,\dotsc,x_s\} \to K, \quad \forall c \in K \setminus \{x_1,\dotsc,x_s\}, \, r(c) = \dfrac{f(c)}{g(c)}
	$$
	где $x_1,\dotsc,x_s$ - корни многочлена $g$.
\end{defn}

\textbf{\uline{Равенство рациональных дробей}}: $\dfrac{f_1}{g_1} = \dfrac{f_2}{g_2} \Leftrightarrow f_1{\cdot}g_2 = f_2{\cdot}g_1$ по правилу пропорции, тогда: 
$$
	\forall c \in  K, \, f_1(c){\cdot}g_2(c) = f_2(c){\cdot}g_1(c) \Rightarrow \dfrac{f_1(c)}{g_1(c)} = \dfrac{f_2(c)}{g_2(c)}, \, g_1(c), g_2(c) \neq 0
$$
Таким образом, если две дроби равны формально, как элементы поля дробей, то они равны и функционально, то есть соответствующие рациональные функции равны. Обратное, как и в случае многочленов, верно не всегда, но для бесконечного поля $K$ обратное будет верно.

\begin{prop}
	Для бесконечного поля $K$, формальное равенство дробей равносильно функциональному.
\end{prop}
\begin{proof}
	Если $\dfrac{f_1(c)}{g_1(c)} = \dfrac{f_2(c)}{g_2(c)}$ при $g_1(c),g_2(c) \neq 0$, тогда:
	$$
		f_1(c){\cdot}g_2(c) = f_2(c){\cdot}g_1(c), \, \forall c \in K \colon g_1(c),g_2(c) \neq 0
	$$
	Поле - бесконечное, а количество корней - конечно $\Rightarrow f_1g_2 - f_2g_1$ имеет бесконечно много корней в $K$, но это возможно лишь в случае, когда это тождественно нулевой многочлен (у ненулевого многочлена число корней не больше его степени), тогда:
	$$
		f_1g_2 - f_2g_1 \equiv 0 \Leftrightarrow f_1g_2 = f_2 g_1 \Leftrightarrow \dfrac{f_1}{g_1} = \dfrac{f_2}{g_2}		
	$$ 
\end{proof}

\newpage
\section*{Структура поля рациональных дробей}
Нашей целью будет изучение структуры рациональных дробей. Применение результатов, которые докажем находят применение в математическом анализе, нежели в алгебре: интегрирование рациональных функций и изучение поведения рациональных функций вблизи их особых точек, то есть там, где они не определены (они ещё называются полюсами).

В связи с такой постановкой задачи возникает вопрос, к какому виду можно привести любую рациональную дробь? Первое упрощение - сократить числитель и знаменатель на общий множитель, то есть привести к виду в котором у числителя и знаменателя нет общих множителей.

\begin{defn}
	Рациональная дробь $\dfrac{f}{g}$ называется \uwave{несократимой}, если $f$ и $g$ взаимно просты.
\end{defn}

\begin{prop}
	$\forall r \in K(x), \, r \neq 0$, можно представить в виде $r = \dfrac{f}{g}$, где $r$ - несократимая дробь, причем $f$ и $g$ \\[4pt]определены однозначно, с точностью до умножения на константу.
\end{prop}
\begin{proof}
	Пусть $r = \dfrac{f}{g}, \, f,g \neq 0$, тогда: $d = (f,g) = \text{НОД}(f,g)$. Положим:
	$$
		f_0 = \dfrac{f}{d}, \, g_0 = \dfrac{g}{d} \Rightarrow f = f_0{\cdot}d,\, g = g_0{\cdot}d \Rightarrow r = \dfrac{f}{g} = \dfrac{f_0{\cdot}d}{g_0{\cdot}d} = \dfrac{f_0}{g_0}	
	$$
	Получили несократимую дробь. Пусть $r = \tfrac{f_1}{g_1} = \tfrac{f_0}{g_0}$ - ещё одна несократимая дробь, тогда по правилу пропорции мы можем написать:
	$$
		\dfrac{f_1}{g_1} = \dfrac{f_0}{g_0} \Leftrightarrow f_0{\cdot}g_1 = f_1{\cdot}g_0
	$$
	Левая часть делится на $f_0 \Rightarrow$ правая часть тоже должна делиться на $f_0$, но $(f_0,g_0) = 1$, тогда $f_1 \divby f_0$:
	$$
		f_1 = f_0{\cdot}h \Rightarrow f_0{\cdot}g_1 = f_1{\cdot}g_0 = f_0{\cdot}h{\cdot}g_0 \Rightarrow g_1 = g_0{\cdot}h
	$$
	Поскольку $(f_1,g_1) = 1$, то никаких общих делителей нет $\Rightarrow h \in K^{\times}$.
\end{proof}

\begin{defn}
	Рациональная дробь $\dfrac{f}{g}$ называется \uwave{правильной}, если $\deg(f) < \deg(g)$ или $f = 0$.
\end{defn}
\begin{prop}
	Правильные дроби образуют подкольцо (без единицы) в $K(x)$.
\end{prop}
\begin{proof}
	Проверим замкнутость относительно сложения и умножения:
	$$
		\forall \, \dfrac{f_0}{g_0}, \dfrac{f_1}{f_1} \in K(x), \, \dfrac{f_0}{g_0}{\cdot}\dfrac{f_1}{g_1} = \dfrac{f_0f_1}{g_0g_1}, \, \deg(f_0f_1) = \deg(f_0) + \deg(f_1) < \deg(g_0) + \deg(g_1) = \deg(g_0g_1)
	$$
	При сложении/вычитании будем считать, что дроби приведены к общему знаменателю:
	$$
		\forall \, \dfrac{f_0}{g}, \dfrac{f_1}{g} \in K(x), \, \dfrac{f_0}{g} \pm \dfrac{f_1}{g} = \dfrac{f_0 \pm f_1}{g}, \, \deg(f_0 \pm f_1) \leq \max\{\deg(f_0),\deg(f_1)\} < \deg(g)
	$$
	Получаем, что множество правильных дробей замкнуто относительно операций сложения и умножения, следовательно, это подкольцо.
\end{proof}
\newpage
\begin{prop}
	$\forall r = \dfrac{f}{g} \in K(x), \, \exists ! \,$ представление дроби: $r = q + \dfrac{h}{g}$, где $q$ - многочлен, а $\dfrac{h}{g}$ - правильная дробь.
\end{prop}
\begin{proof}
	Поделим $f$ на $g$ с остатком, тогда:
	$$
		f = g{\cdot}q + h, \, \deg(h) < \deg(g)
	$$
	Поделим это равенство на $g$ уже в поле дробей, тогда:
	$$
		\dfrac{f}{g} = q + \dfrac{h}{g}
	$$
	где $q$ это многочлен, а $\dfrac{h}{g}$ - правильная дробь, так как $\deg(h) < \deg(g)$. Единственность такого представления следует из единственности деления с остатком. Можно дополнительно проверить пусть у нас есть два таких разложения:
	$$
		q + \dfrac{h}{g} = q_1 + \dfrac{h_1}{g_1} \Rightarrow q - q_1 = \dfrac{h_1}{g_1} - \dfrac{h}{g} = \dfrac{h_2}{g_2}
	$$
	то есть снова получили правильную дробь. Тогда: $(q - q_1){\cdot}g_2 = h_2$, следовательно:
	$$
		\deg((q - q_1)g_2) = \deg(q - q_1) + \deg(g_2) \geq \deg(g_2) > \deg(h_2)
	$$
	Получаем противоречие, за исключением случая, когда $h_2 = 0 \Rightarrow q = q_1 \Rightarrow \dfrac{h_1}{g_1} = \dfrac{h}{g}$.
\end{proof}
Остается вопрос, что делать с правильными дробями? 
\begin{defn}
	\uwave{Простейшая дробь} - это рациональная дробь вида: $\dfrac{h}{p^k}$, где $p$ - это неприводимый многочлен, где $k \in \MN$ и верно: $\deg(h) < \deg(p)$.
\end{defn}
Класс простейших дробей зависит от того поля, над которым мы все рассматриваем.

\textbf{Примеры простейших дробей}:
\begin{enumerate}[label=\arabic*)]
	\item $K = \MC \Rightarrow$ в знаменателе должна стоять степень неприводимого многочлена $\Rightarrow$ степени линейного двухчлена, тогда простейшие дроби имеют вид:
	$$
		\dfrac{a}{(x - z_0)^k}, \, a,z_0 \in \MC
	$$
	\item $K = \MR \Rightarrow$ в знаменателе должна стоять степень неприводимого многочлена $\Rightarrow$ либо степени линейного двухчлена, либо степени квадратного трехчлена без действительных корней, тогда простейшие дроби имеют вид:
	$$
		\dfrac{a}{(x - x_0)^k}, \, a,x_0 \in \MR, \quad \dfrac{dx + e}{(x^2  + bx + c)^l}, \, b^2 - 4c < 0, \, d,e,b,c \in \MR
	$$
\end{enumerate}
\newpage
\subsection*{Разложение правильной дроби в простейшие дроби}
\begin{theorem}
	Пусть $r = \dfrac{f}{g}\in K(x)$ - правильная дробь. Представим знаменатель дроби в виде:
	$$
		g= p_1^{k_1}{\cdot}p_2^{k_2}{\cdot}\dotsc{\cdot}p_s^{k_s}
	$$
	где $p_1,p_2,\dotsc, p_s$ - неприводимые, попарно не пропорциональны. Тогда $\exists!$ разложение правильной дроби в сумму простейших дробей следующего вида:
	$$
		r = \dfrac{h_{11}}{p_1} + \dfrac{h_{12}}{p_1^2} + \dotsc+ \dfrac{h_{1k_1}}{p_1^{k_1}} + \dfrac{h_{21}}{p_2} + \dfrac{h_{22}}{p_2^2} + \dotsc+ \dfrac{h_{2k_2}}{p_2^{k_2}} + \dotsc  + \dfrac{h_{s1}}{p_s} + \dfrac{h_{s2}}{p_s^2} + \dotsc+ \dfrac{h_{sk_s}}{p_s^{k_s}} = \ddsum{\substack{i = 1,\dotsc,s\\k =1,\dotsc,k_i}}{} \dfrac{h_{ik}}{p_i^k}
	$$
\end{theorem}
\begin{proof}\hfill\\
	\textbf{\uline{Существование}}: Существование представления в требуемом виде следует из серии лемм.
	\begin{lemma}
		Если $g = g_1{\cdot}g_2$, где $(g_1,g_2) = 1$, то $\dfrac{f}{g} = \dfrac{f_1}{g_1} + \dfrac{f_2}{g_2}$, где все дроби - правильные.
	\end{lemma}
	\begin{proof}
		Так как $(g_1,g_2) = 1$, тогда: $(g_1,g_2) = 1 = u_1{\cdot}g_1 + u_2 {\cdot}g_2$, где $u_1,u_2$ - многочлены. Домножим на это выражение нашу исходную дробь:
		$$
			\dfrac{f}{g} = \dfrac{f}{g}{\cdot}1 = \dfrac{f{\cdot}(u_1{\cdot}g_1 + u_2 {\cdot}g_2)}{g_1g_2} = \dfrac{f{\cdot}u_2}{g_1} + \dfrac{f{\cdot}u_1}{g_2}
		$$
		Поделим с остатком:
		$$
			f{\cdot}u_1 = g_1{\cdot}q_1 + f_1, \quad f{\cdot}u_2 = g_2{\cdot}q_2 + f_2, \quad \deg(f_i) < \deg(g_i), \, i =1,2 \Rightarrow
		$$
		$$
			\Rightarrow \dfrac{f{\cdot}u_2}{g_1} + \dfrac{f{\cdot}u_1}{g_2} = q_1 + \dfrac{f_1}{g_1} + q_2 + \dfrac{f_2}{g_2} = \underbrace{(q_1 + q_2)}_{\text{многочлен}} + \underbrace{\dfrac{f_1}{g_1} + \dfrac{f_2}{g_2}}_{\text{правильные дроби}}
		$$
		Сумма правильных дробей даёт правильную дробь, исходная дробь также была правильной $\Rightarrow$ поскольку представление дроби в виде суммы правильной дроби и многочлена - единственно, то $q_1 + q_2 = 0$.
	\end{proof}
	\begin{corollary}
		$\exists$ разложение $\dfrac{f}{g} = \dfrac{f_1}{p_1^{k_1}} + \dfrac{f_2}{p_2^{k_2}} + \dotsc + \dfrac{f_s}{p_s^{k_s}}$, где все слагаемые - правильные дроби.
	\end{corollary}
	\begin{proof}
		Индукция по $s$.
		
		\uline{База индукции}: $s = 1 \Rightarrow g = p_1^{k_1} \Rightarrow \dfrac{f}{g} = \dfrac{f}{p_1^{k_1}}$.
		
		\uline{Шаг индукции}: По лемме $1$, нашу дробь можно представить так:
		$$
			\dfrac{f}{g} = \dfrac{f_1}{p_1^{k_1}} + \dfrac{\wte{f}}{p_2^{k_2}{\cdot}\dotsc{\cdot}p_s^{k_s}}
		$$
		где обе дроби - правильные. Тогда по предположению индукции будет верно:
		$$
			\dfrac{f}{g} = \dfrac{f_1}{p_1^{k_1}} + \dfrac{f_2}{p_2^{k_2}} + \dotsc + \dfrac{f_s}{p_s^{k_s}}
		$$
	\end{proof}
	\begin{lemma}
		Правильная дробь $\dfrac{f}{p^k}$, где $p$ - неприводим, может быть разложена в виде суммы простейших:
		$$
			\dfrac{f}{p^k} = \dfrac{h_1}{p} + \dfrac{h_2}{p^2} + \dotsc + \dfrac{h_k}{p^k}, \, \deg(h_i) < \deg(p), \, i = \ovl{1,k}
		$$
	\end{lemma}
	\begin{proof}
		Индукцией по $k$.
		
		\uline{База индукции}: $k = 1 \Rightarrow \dfrac{f}{p} = \dfrac{h_1}{p}$, поскольку исходная дробь правильная, то $\deg(f) = \deg(h_1) < \deg(p) \Rightarrow$ это простейшая дробь.
		
		\uline{Шаг индукции}: Поделим числитель нашей дроби с остатком на $p$:
		$$
			f = p{\cdot}\wte{f} + h_k, \, \deg(h_k) < \deg(p) \Rightarrow \dfrac{f}{p^k} = \dfrac{\wte{f}}{p^{k-1}} + \dfrac{h_k}{p^k}
		$$
		Второе слагаемое будет простейшей дробью, первое - правильной дробью, поскольку:
		$$
			\deg(p\wte{f}) = \deg(\wte{f}) + \deg(p) = \deg(f - h_k) \leq \deg(f) < \deg(p^k) \Rightarrow  \deg(\wte{f}) < \deg(p^{k-1})
		$$
		или просто как разница правильных дробей (одна из которых - простейшая). Применим предположение индукции, тогда:
		$$
			\dfrac{f}{p^k} = \dfrac{\wte{f}}{p^{k-1}} + \dfrac{h_k}{p^k} = \dfrac{h_1}{p} + \dfrac{h_2}{p^2} + \dotsc + \dfrac{h_k}{p^k}
		$$
		где все дроби - простейшие.
	\end{proof}
	Следовательно, мы представляем исходную дробь в виде разложения из следствия $1$ и затем каждую раскладываем по лемме $2$ в виде суммы простейших $\Rightarrow$ существование разложения доказано.
	
	\textbf{\uline{Единственность}}: Предположим, что есть два разложения:
	$$
		\dfrac{f}{g} = \ddsum{\substack{i = 1,\dotsc,s\\k =1,\dotsc,k_i}}{} \dfrac{h_{ik}}{p_i^k} = \ddsum{\substack{i = 1,\dotsc,s\\k =1,\dotsc,k_i}}{} \dfrac{\wte{h}_{ik}}{p_i^k}
	$$
	Предположим, что $\exists \, i,k \colon h_{ik} \neq \wte{h}_{ik}$. Домножим это равенство на $g = p_1^{k_1}{\cdot}p_2^{k_2}{\cdot}\dotsc{\cdot}p_s^{k_s}$, тогда:
	$$
		\ddsum{\substack{i = 1,\dotsc,s\\k =1,\dotsc,k_i}}{} h_{ik}{\cdot}p_1^{k_1}{\cdot}\dotsc{\cdot}p_i^{k_i -k}{\cdot}\dotsc{\cdot}p_s^{k_s} = \ddsum{\substack{i = 1,\dotsc,s\\k =1,\dotsc,k_i}}{} \wte{h}_{ik}{\cdot}p_1^{k_1}{\cdot}\dotsc{\cdot}p_i^{k_i -k}{\cdot}\dotsc{\cdot}p_s^{k_s} \Rightarrow
	$$
	$$
		\Rightarrow  \ddsum{\substack{i = 1,\dotsc,s\\k =1,\dotsc,k_i}}{}(h_{ik} - \wte{h}_{ik}){\cdot}p_1^{k_1}{\cdot}\dotsc{\cdot}p_i^{k_i -k}{\cdot}\dotsc{\cdot}p_s^{k_s} = 0 
	$$
	Выберем $i = i_0, \, k = k_0$ так, чтобы $h_{i_0 k_0} \neq \wte{h}_{i_0 k_0}$, но $h_{i_0 k} = \wte{h}_{i_0 k}$ при $k > k_0$. Тогда:
	$$
		(h_{i_0k_0} - \wte{h}_{i_0k_0}){\cdot}p_1^{k_1}{\cdot}\dotsc{\cdot}p_{i_0}^{k_{i_0} -k_0}{\cdot}\dotsc{\cdot}p_s^{k_s} \neq 0, \ndivby p_{i_0}^{k_{i_0} - k_0 + 1}
	$$
	Отдельно заметим, что:
	$$
		\deg(h_{i_0k_0} - \wte{h}_{i_0k_0}) < \deg(p_{i_0})
	$$
	и соответственно не делится на $p_{i_0}$. Остальные слагаемые в сумме делятся на $p_{i_0}^{k_{i_0} - k_0 + 1}$ поскольку при $i = i_0$ верно либо $k < k_0$ и тогда: $k_{i_0} - k \geq k_{i_0} - k_0 + 1$, либо $k > k_0$ и тогда слагаемое равно $0$. Если $i \neq i_0$, то ничего из степени $p_{i_0}^{k_{i_0}}$ не вычитается и делится на $p_{i_0}^{k_{i_0} - k_0 + 1}$. Следовательно, мы получаем сумму равную нулю, в которой одно слагаемое на что-то не делится, а остальные слагаемые на это делятся $\Rightarrow$ противоречие.
\end{proof}

\section*{Многочлены от нескольких переменных}
Пусть $K$ - коммутативное, ассоциативное кольцо с единицей. Дадим аксиоматическое определение.
\begin{defn}
	\uwave{Кольцо многочленов от $n$ переменных} с коэффициентами из кольца $K$ это ассоциативное, коммутативное кольцо с единицей, удовлетворяющее следующим свойствам:
	\begin{enumerate}[label=\arabic*)]
		\item $K \subset K[x_1,\dotsc, x_n]$;
		\item $x_1,\dotsc,x_n \in K[x_1,\dotsc, x_n], \, x_1,\dotsc, x_n \not\in K$;
		\item $\forall f \in K[x_1,\dotsc,x_n], \, \exists!$ представление: 
		$$
			f = \ddsum{k_1,\dotsc,k_n \geq 0}{}a_{k_1\dotsc k_n}{\cdot}x_1^{k_1}{\cdot}\dotsc{\cdot}x_n^{k_n}
		$$ 
		С точностью до добавления нулевых слагаемых;
	\end{enumerate}
	\textbf{\uwave{Обозначение}}: $K[x_1,\dotsc, k_n]$.
\end{defn}
\begin{defn}
	Элементы $x_1,\dotsc,x_n \in K[x_1,\dotsc, x_n]$ называются \uwave{переменными}.
\end{defn}
\begin{defn}
	Элементы $a_{k_1\dotsc k_n} \in K$ в представлении $f \in K[x_1,\dotsc,x_n]$ называются \uwave{коэффициентами}.
\end{defn}
\begin{defn}
	Элементы $x_1^{k_1}{\cdot}\dotsc{\cdot}x_n^{k_n}$ в представлении $f \in K[x_1,\dotsc,x_n]$ называются \uwave{одночленами}.
\end{defn}
\begin{rem}
	Свойство $3)$ определения кольца многочленов от нескольких переменных можно переформулировать так: любой элемент кольца многочленов от $n$ переменных представляется единственным способом, с точностью до добавления нулевых слагаемых, в виде линейной комбинации одночленов от этих переменных с коэффициентами из исходного кольца $K$.
\end{rem}

\begin{theorem}
	$K[x_1,\dotsc,x_n]$ - существует и единственно с точностью до изоморфизма.
\end{theorem}


\end{document}