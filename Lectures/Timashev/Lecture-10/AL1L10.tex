\documentclass[12pt]{article}
\usepackage[left=1cm, right=1cm, top=2cm,bottom=1.5cm]{geometry} 

\usepackage[parfill]{parskip}
\usepackage[utf8]{inputenc}
\usepackage[T2A]{fontenc}
\usepackage[russian]{babel}
\usepackage{enumitem}
\usepackage[normalem]{ulem}
\usepackage{amsfonts, amsmath, amsthm, amssymb, mathtools,xcolor}
\usepackage{blkarray}

\usepackage{tabularx}
\usepackage{hhline}

\usepackage{accents}
\usepackage{fancyhdr}
\pagestyle{fancy}
\renewcommand{\headrulewidth}{1.5pt}
\renewcommand{\footrulewidth}{1pt}

\usepackage{graphicx}
\usepackage[figurename=Рис.]{caption}
\usepackage{subcaption}
\usepackage{float}

%%Наименование папки откуда забирать изображения
\graphicspath{ {./images/} }

%%Изменение формата для ввода доказательства
\renewcommand{\proofname}{$\square$  \nopunct}
\renewcommand\qedsymbol{$\blacksquare$}

%%Изменение отступа на таблицах
\addto\captionsrussian{%
	\renewcommand{\proofname}{$\square$ \nopunct}%
}
%% Римские цифры
\newcommand{\RN}[1]{%
	\textup{\uppercase\expandafter{\romannumeral#1}}%
}

%% Для удобства записи
\newcommand{\MR}{\mathbb{R}}
\newcommand{\MC}{\mathbb{C}}
\newcommand{\MQ}{\mathbb{Q}}
\newcommand{\MN}{\mathbb{N}}
\newcommand{\MZ}{\mathbb{Z}}
\newcommand{\MTB}{\mathbb{T}}
\newcommand{\MTI}{\mathbb{I}}
\newcommand{\MI}{\mathrm{I}}
\newcommand{\MCI}{\mathcal{I}}
\newcommand{\MJ}{\mathrm{J}}
\newcommand{\MH}{\mathrm{H}}
\newcommand{\MT}{\mathrm{T}}
\newcommand{\MA}{\mathcal{A}}
\newcommand{\MCB}{\mathcal{B}}
\newcommand{\MCC}{\mathcal{C}}
\newcommand{\MCE}{\mathcal{E}}
\newcommand{\MU}{\mathcal{U}}
\newcommand{\MV}{\mathcal{V}}
\newcommand{\MB}{\mathcal{B}}
\newcommand{\MF}{\mathcal{F}}
\newcommand{\MW}{\mathcal{W}}
\newcommand{\ML}{\mathcal{L}}
\newcommand{\MP}{\mathcal{P}}
\newcommand{\VN}{\varnothing}
\newcommand{\VE}{\varepsilon}

\theoremstyle{definition}
\newtheorem{defn}{Опр:}
\newtheorem{rem}{Rm:}
\newtheorem{prop}{Утв.}
\newtheorem{exrc}{Упр.}
\newtheorem{problem}{Задача}
\newtheorem{lemma}{Лемма}
\newtheorem{theorem}{Теорема}
\newtheorem{corollary}{Следствие}

\newenvironment{cusdefn}[1]
{\renewcommand\thedefn{#1}\defn}
{\enddefn}

\DeclareRobustCommand{\divby}{%
	\mathrel{\text{\vbox{\baselineskip.65ex\lineskiplimit0pt\hbox{.}\hbox{.}\hbox{.}}}}%
}
%Короткий минус
\DeclareMathSymbol{\SMN}{\mathbin}{AMSa}{"39}
%Длинная шапка
\newcommand{\overbar}[1]{\mkern 1.5mu\overline{\mkern-1.5mu#1\mkern-1.5mu}\mkern 1.5mu}
%Функция знака
\DeclareMathOperator{\sgn}{sgn}

%Функция ранга
\DeclareMathOperator{\rk}{\text{rk}}
\DeclareMathOperator{\diam}{\text{diam}}


%Обозначение константы
\DeclareMathOperator{\const}{\text{const}}

\DeclareMathOperator{\codim}{\text{codim}}

\DeclareMathOperator*{\dsum}{\displaystyle\sum}
\newcommand{\ddsum}[2]{\displaystyle\sum\limits_{#1}^{#2}}

%Интеграл в большом формате
\DeclareMathOperator{\dint}{\displaystyle\int}
\newcommand{\ddint}[2]{\displaystyle\int\limits_{#1}^{#2}}
\newcommand{\ssum}[1]{\displaystyle \sum\limits_{n=1}^{\infty}{#1}_n}

\newcommand{\smallerrel}[1]{\mathrel{\mathpalette\smallerrelaux{#1}}}
\newcommand{\smallerrelaux}[2]{\raisebox{.1ex}{\scalebox{.75}{$#1#2$}}}

\newcommand{\smallin}{\smallerrel{\in}}
\newcommand{\smallnotin}{\smallerrel{\notin}}

\newcommand*{\medcap}{\mathbin{\scalebox{1.25}{\ensuremath{\cap}}}}%
\newcommand*{\medcup}{\mathbin{\scalebox{1.25}{\ensuremath{\cup}}}}%

\makeatletter
\newcommand{\vast}{\bBigg@{3.5}}
\newcommand{\Vast}{\bBigg@{5}}
\makeatother

%Промежуточное значение для sup\inf, поскольку они имеют разную высоту
\newcommand{\newsup}{\mathop{\smash{\mathrm{sup}}}}
\newcommand{\newinf}{\mathop{\mathrm{inf}\vphantom{\mathrm{sup}}}}

%Скалярное произведение
\newcommand{\inner}[2]{\left\langle #1, #2 \right\rangle }
\newcommand{\linsp}[1]{\left\langle #1 \right\rangle }
\newcommand{\linmer}[2]{\left\langle #1 \vert #2\right\rangle }

%Подпись символов снизу
\newcommand{\ubar}[1]{\underaccent{\bar}{#1}}

%% Шапка для букв сверху
\newcommand{\wte}[1]{\widetilde{#1}}
\newcommand{\wht}[1]{\widehat{#1}}

%%Трансформация Фурье
\newcommand{\fourt}[1]{\mathcal{F}\left(#1\right)}
\newcommand{\ifourt}[1]{\mathcal{F}^{-1}\left(#1\right)}

%%Символ вектора
\newcommand{\vecm}[1]{\overrightarrow{#1\,}}

%%Пространстов матриц
\newcommand{\matsq}[1]{\operatorname{Mat}_{#1}}
\newcommand{\mat}[2]{\operatorname{Mat}_{#1, #2}}


%%Взятие в скобки, модули и норму
\newcommand{\parfit}[1]{\left( #1 \right)}
\newcommand{\modfit}[1]{\left| #1 \right|}
\newcommand{\sqparfit}[1]{\left\{ #1 \right\}}
\newcommand{\normfit}[1]{\left\| #1 \right\|}

%%Функция для обозначения равномерной сходимости по множеству
\newcommand{\uconv}[1]{\overset{#1}{\rightrightarrows}}
\newcommand{\uconvm}[2]{\overset{#1}{\underset{#2}{\rightrightarrows}}}


%%Функция для обозначения нижнего и верхнего интегралов
\def\upint{\mathchoice%
	{\mkern13mu\overline{\vphantom{\intop}\mkern7mu}\mkern-20mu}%
	{\mkern7mu\overline{\vphantom{\intop}\mkern7mu}\mkern-14mu}%
	{\mkern7mu\overline{\vphantom{\intop}\mkern7mu}\mkern-14mu}%
	{\mkern7mu\overline{\vphantom{\intop}\mkern7mu}\mkern-14mu}%
	\int}
\def\lowint{\mkern3mu\underline{\vphantom{\intop}\mkern7mu}\mkern-10mu\int}

%%След матрицы
\DeclareMathOperator*{\tr}{tr}

\makeatletter
\renewcommand*\env@matrix[1][*\c@MaxMatrixCols c]{%
	\hskip -\arraycolsep
	\let\@ifnextchar\new@ifnextchar
	\array{#1}}
\makeatother


%% Переопределение функции хи, чтобы выглядела более приятно
\makeatletter
\@ifdefinable\@latex@chi{\let\@latex@chi\chi}
\renewcommand*\chi{{\@latex@chi\smash[t]{\mathstrut}}} % want only bottom half of \mathstrut
\makeatletter

\begin{document}
\lhead{Алгебра-\RN{1}}
\chead{Тимашев Д.А.}
\rhead{Лекция - 10}
\section*{Применение теории определителей}
	
\subsection*{Нахождение обратной матрицы}

Пусть $A$ - произвольная квадратная  матрица:
$$
	A = 
	\begin{pmatrix}
		a_{11} & a_{12} & \dotsc & a_{1n}\\
		a_{21} & a_{22} & \dotsc & a_{2n}\\
		\vdots & \vdots & \ddots& \vdots \\
		a_{n1} & a_{n2} & \dotsc & a_{nn}
	\end{pmatrix}
$$

\begin{defn}
	\uwave{Присоединенной матрицей} к $A$ называется матрица из алгебраических дополнений вида:
	$$
		\wht{A} = 
		\begin{pmatrix}
			A_{11} & A_{21} & \dotsc & A_{n1}\\
			A_{12} & A_{22} & \dotsc & A_{n2}\\
			\vdots & \vdots & \ddots & \vdots \\
			A_{1n} & A_{2n} & \dotsc & A_{nn} 
		\end{pmatrix}
	$$
	где $\wht{a}_{ij} = A_{ji}$ - алгебраическое дополнение к симметричному элементу $a_{ji} = a^T_{ij}$ исходной матрицы $A$.
\end{defn}

\begin{theorem}
	Пусть $A$ - обратима, тогда будет верно:
	$$
		A^{-1} = \dfrac{1}{\det{A}}{\cdot}\wht{A}
	$$
\end{theorem}
\begin{proof}
	Рассмотрим матрицу $A{\cdot}\wht{A}$, её элемент на месте $(i,j)$:
	$$
		a_{i1}{\cdot}\wht{a}_{1j} + a_{i2}{\cdot}\wht{a}_{2j} + \dotsc + a_{in}{\cdot}\wht{a}_{nj} = a_{i1}{\cdot}A_{j1} + a_{i2}{\cdot}A_{j2} + \dotsc + a_{in}{\cdot}A_{jn} 
	$$
	Если $i = j$, то эта сумма есть разложение по $i$-ой строке определителя матрицы $A$ и равна $\det{A}$. Если же $i \neq j$, то эта сумма есть фальшивое разложение определителя и равна $0$. Тогда:
	$$
		A{\cdot}\wht{A} = 
		\begin{pmatrix}
			\det{A} & 0 & \dotsc & 0\\
			0 & \det{A} & \dotsc & 0\\
			\vdots & \vdots & \ddots & \vdots \\
			0 & 0 & \dotsc & \det{A}
		\end{pmatrix} = \det{A}{\cdot}E
	$$
	Аналогично для $\wht{A}{\cdot}A$, её элемент на месте $(i,j)$:
	$$
		\wht{a}_{i1}{\cdot}a_{1j} + \wht{a}_{i2}{\cdot}a_{2j} + \dotsc + \wht{a}_{in}{\cdot}a_{nj} = A_{1i}{\cdot}a_{1j} + A_{2i}{\cdot}a_{2j} + \dotsc + A_{ni}{\cdot}a_{nj} = 
		\left\{
			\begin{array}{rl}
				\det{A}, & i = j\\
				0, & i \neq j
			\end{array}
		\right.
	$$
	где опять, если $i=j$, то получаем разложение по $i$-ому столбцу, если $i\neq j$, то получаем фальшивое разложение по столбцу. Тогда:
	$$
		\wht{A}{\cdot}A = 		
		\begin{pmatrix}
			\det{A} & 0 & \dotsc & 0\\
			0 & \det{A} & \dotsc & 0\\
			\vdots & \vdots & \ddots & \vdots \\
			0 & 0 & \dotsc & \det{A}
		\end{pmatrix} = \det{A}{\cdot}E
	$$
	Пусть $A$ - обратима $\Rightarrow$ невырождена $\Rightarrow \det{A} \neq 0$. Тогда, каждую формулу поделим на $\det{A}$ и получим:
	$$
		A{\cdot}\dfrac{\wht{A}}{\det{A}} = \dfrac{\wht{A}}{\det{A}}{\cdot}A = E \Rightarrow A^{-1} = \dfrac{\wht{A}}{\det{A}}
	$$
\end{proof}

\textbf{Пример}: пусть $A$ - квадратная матрица $2 \times 2$, тогда:
$$
	A = \begin{pmatrix}
			a & b \\
			c & d
		\end{pmatrix} \Rightarrow 
		\wht{A} = 
		\begin{pmatrix}
			d & -b \\
			-c & a
		\end{pmatrix} \Rightarrow |A| = ad - bc \Rightarrow A^{-1} = 
		\begin{pmatrix}
			\tfrac{d}{ad - bc} & \tfrac{-b}{ad - bc}\\
			\tfrac{-c}{ad - bc} & \tfrac{a}{ad - bc}
		\end{pmatrix}
$$

\subsection*{Метод Крамера}

Рассмотрим квадратную СЛУ (число уравнений $=$ числу неизвестных):
$$
	\left\{
		\begin{matrix}
			a_{11}x_1 & + & \dotsc & + & a_{1n}x_n & = & b_1 \\
			a_{21}x_1 & + & \dotsc & + & a_{2n}x_n & = & b_2 \\
			\vdots & \vdots & \ddots & \vdots & \vdots & \vdots & \vdots \\
			a_{n1}x_1 & + & \dotsc & + & a_{nn}x_n & = & b_n \\
		\end{matrix}
	\right.
$$
Запишем её в матричном виде:
$$
	A{\cdot}x = b, \, A = 
	\begin{pmatrix}
		a_{11} & \dotsc & a_{1n}\\
		\vdots & \ddots & \vdots \\
		a_{n1} & \dotsc & a_{nn}
	\end{pmatrix}, \, x = 
	\begin{pmatrix}
		x_1 \\
		\vdots \\
		x_n
	\end{pmatrix}, \, 
	b = \begin{pmatrix}
		b_1 \\
		\vdots \\
		b_n
	\end{pmatrix}
$$
Мы умеем решать системы методом Гаусса, правда у него есть большой недостаток в том, что он не дает явных формул. Метод решения квадратной СЛУ, называемый \uwave{методом Крамера}, позволяет дать явные формулы для решения квадратной СЛУ. Введем обозначения:
$$
	\Delta = |A| = \det{A}, \quad \Delta_j = 
	\begin{vmatrix}
		a_{11} & a_{12} & \dotsc & a_{1j-1} & b_1 & a_{1 j+ 1} & \dotsc & a_{1n}\\
		a_{21} & a_{22} & \dotsc & a_{2j-1} & b_2 & a_{2 j+ 1} & \dotsc & a_{2n}\\
		\vdots & \vdots & \ddots & \vdots   & \vdots& \vdots   & \ddots & \vdots \\
		a_{n1} & a_{n2} & \dotsc & a_{nj-1} & b_n & a_{n j+ 1} & \dotsc & a_{nn}\\
	\end{vmatrix}, \, \forall j = \overline{1,n}
$$
\begin{prop}(\textbf{правило Крамера})
	\begin{enumerate}[label=\arabic*)]
		\item \textbf{Критерий определенности}: Квадратная СЛУ определена $\Leftrightarrow \Delta \neq 0$;
		\item \textbf{Условие несовместности}: Если $\Delta = 0$, но $\exists \, j \in \{1,\dotsc,n\}\colon \Delta_j \neq 0$, то тогда система несовместна;
		\item \textbf{Формулы Крамера}: В случае, если СЛУ определена (т.е. имеет единственное решение) её единственное решение имеет вид:
		$$
			x_j = \dfrac{\Delta_j}{\Delta}, \, \forall j = \overline{1,n}
		$$
		эти формулы называются \uwave{формулами Крамера};
	\end{enumerate}
\end{prop}
\begin{rem}
	В случае если $\Delta = 0$ и $\forall j =\overline{1,n}, \, \Delta_j = 0$ правило Крамера ничего не говорит.
\end{rem}
\begin{proof}
	Заметим, что $A$ это матрица $n\times n$, а расширенная матрица $\wte{A}$ это матрица $n \times n+1$, тогда мы получаем:
	$$
		\rk{A} \leq \rk{\wte{A}}\leq n
	$$
	\begin{enumerate}[label=\arabic*)]
		\item \textbf{Критерий определенности}: у нас есть этот критерий в терминах рангов:
		$$
			\text{СЛУ определена} \Leftrightarrow \rk{A} = \rk{\wte{A}} = n \Leftrightarrow \rk{A}
		$$
		где последнее следует из нашего замечания выше. Мы знаем, что $\rk{A} = n$ означает, что матрица невырождена, тогда:
		$$
			\rk{A} = n \Leftrightarrow \det{A} = \Delta \neq 0
		$$
		\item \textbf{Условие несовместности}: по условию $\Delta = 0 \Rightarrow \rk{A} < n$, вместе с этим $\exists \, j \in \{1,\dotsc,n\}\colon \Delta_j \neq 0$, тогда соответствующая матрица невырождена $\Rightarrow$ её столбцы линейно независимы:
		$$
			\rk{\left\{A^{(1)}, \dotsc, A^{(j-1)}, b, A^{(j+1)}, \dotsc, A^{(n)}\right\}} = n 
		$$
		Если мы добавим к этой системе $j$-ый столбец матрицы $A$, то ранг может только увеличиться:
 		$$
			\rk{\left\{A^{(1)}, \dotsc, A^{(j-1)}, b, A^{(j+1)}, \dotsc, A^{(n)}\right\}} \leq \rk{\left\{A^{(1)}, \dotsc, A^{(j)}, \dotsc, A^{(n)}, b\right\}} = \rk{\wte{A}} \Rightarrow \rk{\wte{A}} \geq n
		$$
		А в силу нашего замечания, ранг расширенной матрицы не может быть больше $n$, тогда: 
		$$
			\rk{\wte{A}} =n > \rk{A}
		$$ 
		Следовательно, по теореме Кронекера-Капелли эта СЛУ несовместна;
		\item \textbf{Формулы Крамера}: по условию $\Delta \neq 0 \Rightarrow \exists \, A^{-1} = \tfrac{\wht{A}}{\det{A}}$, тогда:
		$$
			Ax = b \Rightarrow A^{-1}{\cdot}Ax = E{\cdot}x = x = A^{-1}{\cdot}b = \dfrac{\wht{A}}{\det{A}}{\cdot}b
		$$
		$$
			\forall j = \overline{1,n}, \, x_j = \dfrac{\wht{a}_{j1}{\cdot}b_1 + \wht{a}_{j2}{\cdot}b_2 + \dotsc + \wht{a}_{jn}{\cdot}b_n}{\Delta} = \dfrac{b_1{\cdot}A_{1j} + b_2{\cdot}A_{2j} + \dotsc + b_n{\cdot}A_{nj}}{\Delta}
		$$
		Заметим, что в числителе записано разложение $\Delta_j$ по $j$-ому столбцу, тогда:
		$$
			\forall j = \overline{1,n}, \, x_j =  \dfrac{b_1{\cdot}A_{1j} + b_2{\cdot}A_{2j} + \dotsc + b_n{\cdot}A_{nj}}{\Delta} = \dfrac{\Delta_j}{\Delta}
		$$
	\end{enumerate}
\end{proof}

\subsection*{Вычисление ранга матрицы}
Рассмотрим определения миноров в произвольной матрице $A$ размера $m \times n$.
\begin{defn}
	\uwave{Минором порядка $k$} матрицы $A$, стоящего на пересечении строк с номерами $i_1,\dotsc,i_k$ и столбцов с номерами $j_1,\dotsc, j_k$, называется определитель соответствующей подматрицы:
	$$
		M_{i_1 \dotsc i_k}^{j_1 \dotsc j_k} = 
		\begin{vmatrix}
			a_{i_1 j_1} & a_{i_1 j_2} & \dotsc & a_{i_1 j_k}\\
			a_{i_2 j_1} & a_{i_2 j_2} & \dotsc & a_{i_2 j_k}\\
			\vdots & \vdots & \ddots & \vdots\\
			a_{i_k j_1} & a_{i_k j_2} & \dotsc & a_{i_k j_k}\\
		\end{vmatrix} = |\wht{A}|, \quad 
		\begin{pmatrix}
			\cdot & \cdot & \dotsc & \cdot  & \cdot& \dotsc & \cdot\\
			\cdot & \color{red}a_{i_1 j_1} & \dotsc & \color{red}a_{i_1 j_2}  & \color{red}a_{i_1 j_k} & \dotsc & \cdot\\
			\cdot & \color{red}a_{i_2 j_1} & \dotsc & \color{red}a_{i_2 j_2}  & \color{red}a_{i_2 j_k} & \dots & \cdot\\
			\vdots & \vdots & \ddots & \vdots  & \vdots& \dotsc & \cdot\\
			\cdot & \color{red}a_{i_k j_1} & \dotsc & \color{red}a_{i_k j_2}  & \color{red}a_{i_k j_k} & \dotsc & \cdot\\
			\vdots & \vdots & \ddots & \vdots  & \vdots& \ddots & \vdots\\
			\cdot & \cdot & \dotsc & \cdot  & \cdot & \dotsc & \cdot
		\end{pmatrix}
	$$
\end{defn}
\begin{defn}
	\uwave{Главным минором порядка $k$} матрицы $A$ называется минор порядка $k$ такой, что номера столбцов и строк соответствующей подматрицы совпадают: 
	$$
		i_1 = j_1, \dotsc, i_k = j_k, \, M_{i_1\dotsc i_k}^{i_1\dotsc i_k} = 	\begin{vmatrix}
			a_{i_1 i_1} & a_{i_1 i_2} & \dotsc & a_{i_1 i_k}\\
			a_{i_2 i_1} & a_{i_2 i_2} & \dotsc & a_{i_2 i_k}\\
			\vdots & \vdots & \ddots & \vdots\\
			a_{i_k i_1} & a_{i_k i_2} & \dotsc & a_{i_k i_k}\\
		\end{vmatrix} , \quad
		\begin{pmatrix}
			\cdot & \cdot & \dotsc & \cdot & \dotsc & \cdot& \dotsc & \cdot\\
			\cdot & \color{red}a_{i_1 i_1} & \dotsc & \color{red}a_{i_1 i_2} & \dotsc & \color{red}a_{i_1 i_k} & \dotsc & \cdot\\
			\vdots & \vdots & \ddots & \vdots & \ddots & \vdots& \ddots & \vdots\\
			\cdot & \color{red}a_{i_2 i_1} & \dotsc & \color{red}a_{i_2 i_2} & \dotsc & \color{red}a_{i_2 i_k} & \dots & \cdot\\
			\vdots & \vdots & \ddots & \vdots & \ddots & \vdots& \dotsc & \cdot\\
			\cdot & \color{red}a_{i_k i_1} & \dotsc & \color{red}a_{i_k i_2} & \dotsc & \color{red}a_{i_k i_k} & \dotsc & \cdot\\
			\vdots & \vdots & \ddots & \vdots & \ddots & \vdots& \ddots & \vdots\\
			\cdot & \cdot & \dotsc & \cdot & \dotsc & \cdot & \dotsc & \cdot
		\end{pmatrix}
	$$
\end{defn}
\begin{defn}
	\uwave{Угловым минором порядка $k$} матрицы $A$ называется главный минор порядка $k$ такой, что номера столбцов и строк соответствующей подматрицы идут по порядку от $1$ до $k$: 
	$$
		i_1 = j_1 = 1, \dotsc, i_k = j_k = k, \, M_{1\dotsc k}^{1\dotsc k} = 	\begin{vmatrix}
			a_{11} & a_{12} & \dotsc & a_{1k}\\
			a_{21} & a_{22} & \dotsc & a_{2k}\\
			\vdots & \vdots & \ddots & \vdots\\
			a_{k1} & a_{k2} & \dotsc & a_{kk}\\
		\end{vmatrix} , \quad
		\begin{pmatrix}
			\color{red}a_{11} & \color{red}a_{12} & \dotsc & \color{red}a_{1k} & \dotsc & \cdot\\
			\color{red}a_{21} & \color{red}a_{22} & \dotsc & \color{red}a_{2k} & \dotsc & \cdot\\
			\vdots & \vdots & \ddots & \vdots & \ddots & \vdots\\
			\color{red}a_{k1} & \color{red}a_{k2} & \dotsc & \color{red}a_{kk} & \dotsc & \cdot\\
			\vdots & \vdots & \ddots & \vdots & \ddots & \vdots\\
			\cdot & \cdot & \dotsc & \cdot & \dotsc &\cdot
		\end{pmatrix}
	$$
\end{defn}

\begin{theorem}(\textbf{о ранге матрицы})
	Ранг произвольной матрицы равен наибольшему порядку её ненулевого минора.
\end{theorem}

\begin{proof}
	Пусть $r$ - наибольший порядок ненулевого минора в $A$:
	$$
		M_{i_1 \dotsc i_r}^{j_1 \dotsc j_r} \neq 0
	$$
	Переставим строки и столбцы в $A$ (такая операция может повлиять на миноры только сменой знака) так, чтобы этот минор был сосредоточен в верхнем левом углу матрицы:
	$$
		\begin{pmatrix}[ccc|ccc]
			& & & a_{1 r+ 1} &\dotsc & a_{1n}\\
			& \overline{A} & & \vdots & \ddots & \vdots \\
			& & & a_{r r + 1} & \dotsc & a_{r n}\\ \hline
			a_{r+1 1} & \dotsc & a_{r + 1 r} & a_{r+1 r + 1} & \dotsc & a_{r+1 n}\\
			\vdots & \ddots & \vdots & \vdots & \ddots & \vdots \\
			a_{m1} & \dotsc & a_{m r} & a_{m r+1} & \dotsc & a_{mn}			 
		\end{pmatrix}, \quad M_{1 \dotsc r}^{1 \dotsc r} = \det{\overline{A}}\neq 0
	$$
	следовательно $\overline{A}$ - невырождена $\Rightarrow$ её строки линейно независимы. Введем обозначение:
	$$
		\forall i = 1,\dotsc,m , \, \overline{A}_i = (a_{i1}, \dotsc , a_{ir}) \in \MR^r
	$$
	Тогда $\overline{A}_1, \dotsc, \overline{A}_r$ - линейно независимы и их $r$ штук $\Rightarrow$ они образуют базис $\MR^r$. Тогда:
	$$
		\forall i > r, \, \exists! \, \lambda_1, \dotsc, \lambda_r \colon \overline{A}_i = \lambda_1 \overline{A}_1 + \dotsc + \lambda_r \overline{A}_r 
	$$
	Строки $A_1,\dotsc,A_r$ тем более линейно независимы, поскольку иначе:
	$$
		\exists \, \mu_1, \dotsc, \mu_r \colon \mu_i \neq 0, \,  \mu_1 A_1 + \dotsc \mu_r A_r = 0 \Rightarrow \mu_1 \overline{A}_1 + \dotsc + \mu_r \overline{A}_r =0  
	$$
	Добавим к первым $r$ строкам и первым $r$ столбцам ещё $i$-ую строку и $j$-ый столбец. Тогда на пересечении этих строк и столбцов мы получим минор порядка $r+1$, называемый \uwave{окаймляющим} для исходного:
	$$
		\begin{pmatrix}[ccc|ccccc]
			& & & a_{1 r+ 1} &\dotsc & a_{1j} & \dotsc & a_{1n}\\
			& \overline{A} & & \vdots & \ddots & \vdots & \ddots& \vdots \\
			& & & a_{r r + 1} & \dotsc & a_{r j} & \dotsc & a_{r n} \\ \hline
			a_{r+1 1} & \dotsc & a_{r + 1 r} & a_{r+1 r + 1} & \dotsc & a_{r + 1j} & \dotsc & a_{r+1 n}\\
			\vdots & \ddots & \vdots & \vdots & \ddots & \vdots & \ddots& \vdots \\
			a_{i1} & \dotsc & a_{i r} & a_{i r+1} & \dotsc & a_{i j}& \dotsc & a_{in}	\\
			\vdots & \ddots & \vdots & \vdots & \ddots & \vdots & \ddots& \vdots \\
			a_{m1} & \dotsc & a_{m r} & a_{m r+1} & \dotsc & a_{m j}& \dotsc & a_{mn}			 
		\end{pmatrix}, \quad \overline{\overline{A}} = 
		\begin{pmatrix}[ccc|c]
			& & & a_{1 j} \\
			& \overline{A} & & \vdots \\
			& & & a_{r j} \\ \hline
			a_{i1} & \dotsc & a_{ir} & a_{ij}
		\end{pmatrix}
	$$
	Он равен нулю в силу того, что наибольший порядок ненулевого минора равен $r$:
	$$
		\forall i,j > r, \, M_{1\dotsc r i}^{1\dotsc r j} = \det{\overline{\overline{A}}} = 0
	$$
	Строки $\overline{\overline{A}}_1, \dotsc, \overline{\overline{A}}_r$ - линейно независимы, иначе строки $\overline{A}_1, \dotsc, \overline{A}_r$ были бы линейно зависимы. При этом все строки $\overline{\overline{A}}_1, \dotsc, \overline{\overline{A}}_r, \overline{\overline{A}}_{r+1}$ - линейно зависимы, так как определитель равен $0$. По свойствам линейной зависимости отсюда следует:
	$$
		\overline{\overline{A}}_{r+1} = \lambda'_1{\cdot}\overline{\overline{A}}_{1} + \dotsc + \lambda'_r{\cdot}\overline{\overline{A}}_{r} \Rightarrow  \overline{A}_{r+1} = \lambda'_1{\cdot}\overline{A}_{1} + \dotsc + \lambda'_r{\cdot}\overline{A}_{r}
	$$
	где последнее верно при отбрасывании последнего столбца. Но в силу однозначности разложения коротких строк по базису, коэффициенты будут такими же, как и в этом разложении:
	$$
		\lambda'_1 = \lambda_1, \dotsc, \lambda'_r = \lambda_r
	$$
	Следовательно, если мы добавляем $i$-ую строку к матрице $\overline{A}$ и любой столбец $j$, то строка $i$ будет выражаться через первые $r$ строк всегда с одними и теми же коэффициентами. Таким образом:
	$$
		\forall j = 1,\dotsc, n, \, a_{ij} = \lambda_1 a_{1j} + \lambda_2 a_{2j} + \dotsc + \lambda_r a_{rj}
	$$
	Это будет верно $\forall j = 1,\dotsc, n$, поскольку $\lambda_1,\dotsc, \lambda_r$ от выбора столбца $j$ не зависят, тогда:
	$$
		\forall i > r, \, A_i = \lambda_1 A_1 + \lambda_2 A_2 + \dotsc + \lambda_r A_r
	$$
	В результате, $\{A_1,\dotsc, A_r\}$ это базис системы строк матрицы $A$ и тогда $\rk{A} = r$.
\end{proof}

\begin{rem}
	Доказательство теоремы показывает, что ранг матрицы равен $r$, если $\exists \, M_{i_1 \dotsc i_r}^{j_1 \dotsc j_r} \neq 0$ такой, что:
	$$
		\forall i \neq i_1, \dotsc, i_r, \, \forall j \neq j_1 ,\dotsc, j_r,\, M_{i_1 \dotsc i_r i}^{j_1 \dotsc j_r j} = 0
	$$
	то есть, все окаймляющие миноры равны нулю.
\end{rem}
Из этого замечания получается метод нахождения ранга матрицы, который теоретически бывает полезен (хотя и не очень полезен на практике).

\subsection*{Метод окаймляющих миноров}

$1)$ Если $A = 0$, то $\rk{A} = 0$.

$2)$ Если $A \neq 0$, то возьмем в ней $a_{i_1 j_1} \neq 0$, это минор первого порядка, отличный от нуля. Рассмотрим ещё какую-нибудь строку $i$ и какой-нибудь столбец $j$. Посмотрим на окаймляющий минор:
$$
	\forall i \neq i_1, \, \forall j \neq j_1, \, M_{i_1 i}^{j_1 j} = ?
$$
Если все такие миноры равны нулю, то $\rk{A} = 1$. Если же нашелся окаймляющий минор не равный нулю, то возьмем его для следующего шага.

$3)$ Обозначим минор с шага $2)$ как $M_{i_1 i_2}^{j_1 j_2} \neq 0$. Рассмотрим ещё какую-нибудь строку $i$ и какой-нибудь столбец $j$. Посмотрим на окаймляющий минор:
$$
	\forall i \neq i_1,i_2, \, \forall j \neq j_1,j_2, \, M_{i_1 i_2 i}^{j_1 j_2 j} = ?
$$
Если все такие миноры равны нулю, то $\rk{A} = 2$. Если же нашелся окаймляющий минор не равный нулю, то возьмем его для следующего шага. И так далее. 

В конце получим ненулевой минор порядка $r$ такой, что все окаймляющие миноры $r+1$-го порядка равны нулю: 
$$
	\forall i \neq i_1,i_2, \, \forall j \neq j_1,j_2, \, M_{i_1 \dotsc i_r i}^{j_1 \dotsc j_r j} =0
$$
Тогда можем сделать вывод, что $\rk{A} = r$.

\end{document}