\documentclass[12pt]{article}
\usepackage[left=1cm, right=1cm, top=2cm,bottom=1.5cm]{geometry} 

\usepackage[parfill]{parskip}
\usepackage[utf8]{inputenc}
\usepackage[T2A]{fontenc}
\usepackage[russian]{babel}
\usepackage{enumitem}
\usepackage[normalem]{ulem}
\usepackage{amsfonts, amsmath, amsthm, amssymb, mathtools,xcolor}
\usepackage{blkarray}

\usepackage{tabularx}
\usepackage{hhline}

\usepackage{accents}
\usepackage{fancyhdr}
\pagestyle{fancy}
\renewcommand{\headrulewidth}{1.5pt}
\renewcommand{\footrulewidth}{1pt}

\usepackage{graphicx}
\usepackage[figurename=Рис.]{caption}
\usepackage{subcaption}
\usepackage{float}

%%Наименование папки откуда забирать изображения
\graphicspath{ {./images/} }

%%Изменение формата для ввода доказательства
\renewcommand{\proofname}{$\square$  \nopunct}
\renewcommand\qedsymbol{$\blacksquare$}

%%Изменение отступа на таблицах
\addto\captionsrussian{%
	\renewcommand{\proofname}{$\square$ \nopunct}%
}
%% Римские цифры
\newcommand{\RN}[1]{%
	\textup{\uppercase\expandafter{\romannumeral#1}}%
}

%% Для удобства записи
\newcommand{\MR}{\mathbb{R}}
\newcommand{\MC}{\mathbb{C}}
\newcommand{\MQ}{\mathbb{Q}}
\newcommand{\MN}{\mathbb{N}}
\newcommand{\MZ}{\mathbb{Z}}
\newcommand{\MTB}{\mathbb{T}}
\newcommand{\MTI}{\mathbb{I}}
\newcommand{\MI}{\mathrm{I}}
\newcommand{\MCI}{\mathcal{I}}
\newcommand{\MJ}{\mathrm{J}}
\newcommand{\MH}{\mathrm{H}}
\newcommand{\MT}{\mathrm{T}}
\newcommand{\MA}{\mathcal{A}}
\newcommand{\MCB}{\mathcal{B}}
\newcommand{\MCC}{\mathcal{C}}
\newcommand{\MCE}{\mathcal{E}}
\newcommand{\MU}{\mathcal{U}}
\newcommand{\MV}{\mathcal{V}}
\newcommand{\MB}{\mathcal{B}}
\newcommand{\MF}{\mathcal{F}}
\newcommand{\MW}{\mathcal{W}}
\newcommand{\ML}{\mathcal{L}}
\newcommand{\MP}{\mathcal{P}}
\newcommand{\VN}{\varnothing}
\newcommand{\VE}{\varepsilon}

\theoremstyle{definition}
\newtheorem{defn}{Опр:}
\newtheorem{rem}{Rm:}
\newtheorem{prop}{Утв.}
\newtheorem{exrc}{Упр.}
\newtheorem{problem}{Задача}
\newtheorem{lemma}{Лемма}
\newtheorem{theorem}{Теорема}
\newtheorem{corollary}{Следствие}

\newenvironment{cusdefn}[1]
{\renewcommand\thedefn{#1}\defn}
{\enddefn}

\DeclareRobustCommand{\divby}{%
	\mathrel{\text{\vbox{\baselineskip.65ex\lineskiplimit0pt\hbox{.}\hbox{.}\hbox{.}}}}%
}
%Короткий минус
\DeclareMathSymbol{\SMN}{\mathbin}{AMSa}{"39}
%Длинная шапка
\newcommand{\overbar}[1]{\mkern 1.5mu\overline{\mkern-1.5mu#1\mkern-1.5mu}\mkern 1.5mu}
%Функция знака
\DeclareMathOperator{\sgn}{sgn}

%Функция ранга
\DeclareMathOperator{\rk}{\text{rk}}
\DeclareMathOperator{\diam}{\text{diam}}


%Обозначение константы
\DeclareMathOperator{\const}{\text{const}}

\DeclareMathOperator{\codim}{\text{codim}}

\DeclareMathOperator*{\dsum}{\displaystyle\sum}
\newcommand{\ddsum}[2]{\displaystyle\sum\limits_{#1}^{#2}}

%Интеграл в большом формате
\DeclareMathOperator{\dint}{\displaystyle\int}
\newcommand{\ddint}[2]{\displaystyle\int\limits_{#1}^{#2}}
\newcommand{\ssum}[1]{\displaystyle \sum\limits_{n=1}^{\infty}{#1}_n}

\newcommand{\smallerrel}[1]{\mathrel{\mathpalette\smallerrelaux{#1}}}
\newcommand{\smallerrelaux}[2]{\raisebox{.1ex}{\scalebox{.75}{$#1#2$}}}

\newcommand{\smallin}{\smallerrel{\in}}
\newcommand{\smallnotin}{\smallerrel{\notin}}

\newcommand*{\medcap}{\mathbin{\scalebox{1.25}{\ensuremath{\cap}}}}%
\newcommand*{\medcup}{\mathbin{\scalebox{1.25}{\ensuremath{\cup}}}}%

\makeatletter
\newcommand{\vast}{\bBigg@{3.5}}
\newcommand{\Vast}{\bBigg@{5}}
\makeatother

%Промежуточное значение для sup\inf, поскольку они имеют разную высоту
\newcommand{\newsup}{\mathop{\smash{\mathrm{sup}}}}
\newcommand{\newinf}{\mathop{\mathrm{inf}\vphantom{\mathrm{sup}}}}

%Скалярное произведение
\newcommand{\inner}[2]{\left\langle #1, #2 \right\rangle }
\newcommand{\linsp}[1]{\left\langle #1 \right\rangle }
\newcommand{\linmer}[2]{\left\langle #1 \vert #2\right\rangle }

%Подпись символов снизу
\newcommand{\ubar}[1]{\underaccent{\bar}{#1}}

%% Шапка для букв сверху
\newcommand{\wte}[1]{\widetilde{#1}}
\newcommand{\wht}[1]{\widehat{#1}}

%%Трансформация Фурье
\newcommand{\fourt}[1]{\mathcal{F}\left(#1\right)}
\newcommand{\ifourt}[1]{\mathcal{F}^{-1}\left(#1\right)}

%%Символ вектора
\newcommand{\vecm}[1]{\overrightarrow{#1\,}}

%%Пространстов матриц
\newcommand{\matsq}[1]{\operatorname{Mat}_{#1}}
\newcommand{\mat}[2]{\operatorname{Mat}_{#1, #2}}


%%Взятие в скобки, модули и норму
\newcommand{\parfit}[1]{\left( #1 \right)}
\newcommand{\modfit}[1]{\left| #1 \right|}
\newcommand{\sqparfit}[1]{\left\{ #1 \right\}}
\newcommand{\normfit}[1]{\left\| #1 \right\|}

%%Функция для обозначения равномерной сходимости по множеству
\newcommand{\uconv}[1]{\overset{#1}{\rightrightarrows}}
\newcommand{\uconvm}[2]{\overset{#1}{\underset{#2}{\rightrightarrows}}}


%%Функция для обозначения нижнего и верхнего интегралов
\def\upint{\mathchoice%
	{\mkern13mu\overline{\vphantom{\intop}\mkern7mu}\mkern-20mu}%
	{\mkern7mu\overline{\vphantom{\intop}\mkern7mu}\mkern-14mu}%
	{\mkern7mu\overline{\vphantom{\intop}\mkern7mu}\mkern-14mu}%
	{\mkern7mu\overline{\vphantom{\intop}\mkern7mu}\mkern-14mu}%
	\int}
\def\lowint{\mkern3mu\underline{\vphantom{\intop}\mkern7mu}\mkern-10mu\int}

%%След матрицы
\DeclareMathOperator*{\tr}{tr}

\makeatletter
\renewcommand*\env@matrix[1][*\c@MaxMatrixCols c]{%
	\hskip -\arraycolsep
	\let\@ifnextchar\new@ifnextchar
	\array{#1}}
\makeatother


%% Переопределение функции хи, чтобы выглядела более приятно
\makeatletter
\@ifdefinable\@latex@chi{\let\@latex@chi\chi}
\renewcommand*\chi{{\@latex@chi\smash[t]{\mathstrut}}} % want only bottom half of \mathstrut
\makeatletter

\begin{document}
\lhead{Алгебра-\RN{1}}
\chead{Тимашев Д.А.}
\rhead{Лекция - 8}
\section*{Теория перестановок и подстановок}
\subsection*{Транспозиции}

\begin{defn}
	\uwave{Транспозиция} это цикл длины $2$: 
	$$
		\tau = (i,j) = (ij), \, \tau(i) = j, \, \tau(j) = i, \, \tau(k) = k,\, \forall k \neq i,j
	$$
\end{defn}
\begin{prop}
	Любая подстановка $\sigma \in S_n$ разлагается в произведение транспозиций. 
\end{prop}
\begin{proof}
	Разложим $\sigma$ в произведение попарно независимых циклов, воспользовавшись предыдущей теоремой:
	$$
		\sigma = \sigma_1{\cdot}\sigma_2{\cdot}\dotsc{\cdot}\sigma_s
	$$
	Достаточно разложить каждый $\sigma_i$ в произведение транспозиций. Можно считать, что исходная подстановка $\sigma = (i_1i_2\dotsc i_l)$ это цикл длины $l$. Тогда этот цикл разлагается в произведение цикла длины $l-1$ и транспозиции двух последних номеров $\sigma$:
	$$
		\sigma = (i_1 i_2 \dotsc i_{l-1}){\cdot}(i_{l-1} i_l )
	$$ 
	Понятно, что все числа, которые не вошли в орбиту $\sigma$ они с помощью $\sigma$ и с помощью подстановки справа остаются на месте. Докажем, что на числа из орбиты $\sigma$ обе подстановки действуют одинаково:
	$$
		\begin{array}{rl}
			\tau = (i_{l-1}i_l) \colon & i_{l-1} \to i_l \to i_{l-1} \\ [5pt]
			\pi = (i_1 i_2 \dotsc i_{l-1}) \colon & i_1 \to i_2 \to i_3 \to \dotsc \to i_{l-1} \to i_1\\ [5pt]
			\pi{\cdot}\tau \colon & i_1 \xrightarrow{\tau} i_1 \xrightarrow{\pi} i_2 \xrightarrow{\tau} i_2 \xrightarrow{\pi} i_3 \xrightarrow{\tau} \dotsc \xrightarrow{\tau} i_{l-2} \xrightarrow{\pi} i_{l-1} \xrightarrow{\tau}  i_l \xrightarrow{\pi} i_l \xrightarrow{\tau} i_{l-1} \xrightarrow{\pi} i_1
		\end{array}
	$$
	Таким образом, подстановка $(i_1 i_2 \dotsc i_{l-1}){\cdot}(i_{l-1} i_l )$ действует на числа $i_1,\dotsc, i_l$ также, как и $\sigma$. Далее, мы можем аналогично разложить цикл длины $l-1$, затем цикл длины $l-2$ и так далее, тогда:
	$$
		\sigma = (i_1 i_2 \dotsc i_{l-1}){\cdot}(i_{l-1} i_l ) = (i_1 i_2 \dotsc i_{l-2}){\cdot}(i_{l-2} i_{l-1} ){\cdot}(i_{l-1} i_l ) = \dotsc = (i_1 i_2){\cdot}(i_2 i_3){\cdot}\dotsc{\cdot}(i_{l-1}i_l)
	$$
	И таким образом, мы можем разложить каждый независимый цикл $\Rightarrow$ получаем требуемое.
\end{proof}
\begin{defn}
	\uwave{Инверсия} (беспорядок) в перестановке $(i_1i_2 \dotsc i_n)$ это пара чисел $(i_k i_l)$, где $k < l$, но $i_k > i_l$.
\end{defn}
\begin{defn}
	Перестановка \uwave{чётна}, если количество инверсий в этой перестановке чётно.
\end{defn}
\begin{defn}
	Перестановка \uwave{нечётна}, если количество инверсий в этой перестановке нечётно.
\end{defn}
\begin{defn}
	\uwave{Знак перестановки} $\sgn{(i_1\dotsc i_l)} = 
	\begin{cases}
		1, & (i_1\dotsc i_l) \text{ - чётна}\\
		-1, & (i_1\dotsc i_l) \text{ - нечётна}
	\end{cases}$.
\end{defn}

\textbf{Пример}: рассмотрим перестановку номеров $\{1,2,3,4, 5\}$: $(25413)$. Число инверсий:
$$
	1 \colon (21), (51), (41), \, 2 \colon 0, \, 3 \colon (53), (43), \, 4 \colon (54), \, 5 \colon 0 \Rightarrow 3 + 0 + 2 + 1 + 0 = 6
$$
то есть перестановка является чётной и её знак равен $1$.
\begin{defn}
	Подстановка $\sigma = \begin{pmatrix}
		1 & 2 & \dotsc & n\\
		i_1 & i_2 & \dotsc & n
	\end{pmatrix}$ называется \uwave{чётной}, если перестановка $(i_1 i_2 \dotsc i_n)$ чётна.
\end{defn}
\begin{defn}
	Подстановка $\sigma = \begin{pmatrix}
		1 & 2 & \dotsc & n\\
		i_1 & i_2 & \dotsc & n
	\end{pmatrix}$ называется \uwave{нечётной}, если перестановка $(i_1 i_2 \dotsc i_n)$ нечётна.
\end{defn}
\newpage
\begin{defn}
	\uwave{Знак подстановки} $\sigma = \begin{pmatrix}
		1 & 2 & \dotsc & n\\
		i_1 & i_2 & \dotsc & n
	\end{pmatrix}$ равен $\sgn{(i_1 i_2 \dotsc i_n)}$.
\end{defn}

\begin{prop}
	При транспозиции двух элементов в перестановке, её четность меняется.
\end{prop}
\begin{proof}
	Рассмотрим перестановку:
	$$
		(i_1, \dotsc, i_k, \dotsc, i_l, \dotsc , i_n)
	$$
	Предположим, что мы переставили местами два элемента: $i_k \leftrightarrow i_l$. Теперь мы получили новую перестановку:
	$$
		(i_1, \dotsc, i_l, \dotsc, i_k, \dotsc , i_n)
	$$
	Выясним число инверсий в новой перестановке. 
	\begin{enumerate}[label=\arabic*)]
		\item Среди пар $(i_p, i_q)$, где $p,q \neq k,l$ число инверсий не меняется, поскольку они остаются на местах;
		
		\item  Среди пар $(i_p, i_k), \, p< k\vee p > l$:
		$$
			(i_1, \dotsc, {\color{red}i_p}, \dotsc, i_k, \dotsc, i_l, \dotsc , i_n) \vee (i_1, \dotsc, i_k, \dotsc, i_l, \dotsc, {\color{red}i_p}, \dotsc, i_n)
		$$
		Если $p < k$, то порядок следования элементов среди $i_p$ не изменился: $i_p$ все ещё идут перед $i_k$. Аналогично для $ p > l$: $i_p$ все ещё идут после $i_k$:
		$$
			(i_1, \dotsc, {\color{red}i_p}, \dotsc, i_l, \dotsc, i_k, \dotsc , i_n) \vee (i_1, \dotsc, i_l, \dotsc, i_k, \dotsc,{\color{red}i_p}, \dotsc, i_n)
		$$
		\item Аналогично, среди пар $(i_p, i_l), \, p< k\vee p > l$ количество инверсии не меняется;
		
		\item Среди пар $(i_q, i_k), \, k < q < l$, количество инверсий меняется на $\pm 1, \pm 1, \dotsc, \pm 1$. При этом слагаемых столько, сколько этих пар $=$ количество чисел $q$ между $k$ и $l$: $l - k -1$;
		$$
			(i_1, \dotsc,  i_k, \underbrace{\dotsc,{\color{red}i_p}, \dotsc}_{l -k -1}, i_l, \dotsc , i_n) 
		$$
		
		\item Аналогично, среди пар $(i_q, i_l), \, k < q < l$, количество инверсий меняется на $\pm 1, \pm 1, \dotsc, \pm 1$. При этом слагаемых столько, сколько этих пар $=$ количество чисел $q$ между $k$ и $l$: $l - k -1$;
		
		\item В паре $(i_k i_l)$ после перестановки либо инверсия появилась, либо она исчезла, если она была;
	\end{enumerate}
	Общее изменение количества инверсий равно: 
	$$ 
		\underbrace{\pm 1, \dotsc, \pm 1}_{l -k - 1} \underbrace{\pm 1, \dotsc, \pm 1}_{l -k - 1} \pm 1 = \underbrace{\pm 1, \dotsc, \pm 1}_{2(l -k - 1) + 1}
	$$ 
	Поскольку $2(l - k - 1) + 1$ - нечетно, то число инверсий изменилось на нечетное число $\Rightarrow$ чётность изменится. 
\end{proof}

\begin{corollary}
	$\#$ чётных и нечётных перестановок (а значит и подстановок) одинаковое и равно $\dfrac{n!}{2}$.
\end{corollary}
\begin{proof}
	Установим взаимнооднозначное соответствие между множествами чётных и нечётных перестановок:
	$$
		(i_1 i_2 i_3 \dotsc i_l) \leftrightarrow (i_2 i_1 i_3 \dotsc i_n)
	$$
	Перестановка первых двух номеров меняет четность перестановки $\Rightarrow$ это взаимнооднозначное отображение между чётными и нечётными перестановками $\Rightarrow$ на каждое множество приходится одинаковое число элементов $\Rightarrow \tfrac{n!}{2}$. 
\end{proof}
\begin{rem}
	Заметим, что взаимооднозначность отображения выше можно легко понять из-за конечности множества перестановок и инъективности отображения.
\end{rem}

\begin{prop}
	Если подстановка $\sigma = \tau_1{\cdot}\tau_2 {\cdot}\dotsc {\cdot}\tau_N$, где $\tau_i$ - транспозиции, то тогда $\sgn{\sigma} = (-1)^N$.
\end{prop}
\begin{proof}
	Индукцией по числу множителей:
	
	\textbf{\uline{База}}: $N = 0 \Rightarrow \sigma = \VE \Rightarrow$ инверсий нет, поскольку верхняя и нижняя строки совпадают $\Rightarrow \sgn{\sigma} = 1$.
	
	\textbf{\uline{Шаг}}: $\sigma = \tau_1{\cdot}\tau_2{\cdot} \dotsc {\cdot}\tau_{N-1} {\cdot}\tau_N = \sigma'{\cdot}\tau_N$, где $\tau_N = (kl)$, запишем $\sigma'$ в стандартной двухрядной записи:
	$$
		\sigma' = \begin{pmatrix}
			1 & \dotsc  & k & \dotsc & l & \dotsc & n \\
			i_1 & \dotsc & i_k & \dotsc & i_l & \dotsc & i_n
		\end{pmatrix} \Rightarrow 
		\sigma = 
		\begin{pmatrix}
			1 & \dotsc & k & \dotsc & l & \dotsc & n\\
			i_1 & \dotsc & i_l & \dotsc & i_k & \dotsc & i_n
		\end{pmatrix}
	$$
	Таким образом, произошла перестановка двух номеров местами. По-доказанному ранее предложению, чётность поменялась: $\sgn{\sigma} = -\sgn{\sigma'}$. По предположению индукции:
	$$
		\sgn{\sigma'} = (-1)^{N-1} \Rightarrow \sgn{\sigma} = -(-1)^{N-1} = (-1)^{N}
	$$
\end{proof}

\begin{prop}
	$\forall \sigma,\sigma' \in S_n, \, \sgn{(\sigma{\cdot}\sigma')} = \sgn{\sigma}{\cdot}\sgn{\sigma'}$.
\end{prop}
\begin{proof}
	Разложим подстановку $\sigma$ и $\sigma'$ в произведение транспозиций:
	$$
		\sigma = \tau_1{\cdot}\dotsc{\cdot}\tau_N, \, \sigma' = \tau'_1{\cdot}\dotsc{\cdot}\tau'_M
	$$
	где $\tau_i, \, \tau'_i$ - транспозиции. Тогда:
	$$
		\sigma{\cdot}\sigma' = \tau_1{\cdot}\dotsc{\cdot}\tau_N{\cdot}\tau'_1{\cdot}\dotsc{\cdot}\tau'_M
	$$
	По доказанному предложению выше, мы получаем:
	$$
		\sgn{(\sigma{\cdot}\sigma')} = (-1)^{N + M} = (-1)^N{\cdot}(-1)^{M} = \sgn{\sigma}{\cdot}\sgn{\sigma'}
	$$
\end{proof}
\begin{prop}
	Знак обратной подстановки равен знаку самой подстановки: $\forall \sigma \in S_n, \, \sgn{(\sigma^{-1})} = \sgn{\sigma}$.
\end{prop}
\begin{proof}
	Рассмотрим произведение $\sigma{\cdot}\sigma^{-1} = \VE$, тогда:
	$$
		\sgn{\sigma}{\cdot}\sgn{\sigma^{-1}} = \sgn{(\sigma{\cdot}\sigma^{-1})} = \sgn{\VE} = 1 
	$$
	Но поскольку $\sgn{\sigma} = \pm1$, $\sgn{\sigma^{-1}} = \pm1$ и их произведение равно $1 \Rightarrow$ они равны.
\end{proof}

\newpage
\section*{Теория определителей}
\begin{defn}
	Пусть $A$ - квадратная матрица размера $n\times n$. Её \uwave{определитель} это число:
	$$
		\det{A} = \begin{vmatrix}
			a_{11} & \dotsc & a_{1n}\\
			\vdots & \ddots & \vdots \\
			a_{n1} & \dotsc & a_{nn}
		\end{vmatrix} =
		\ddsum{(i_1i_2 \dotsc i_n)}{}a_{1{i_1}}{\cdot}a_{2{i_2}}{\cdot}\dotsc{\cdot}a_{n{i_n}}{\cdot}\sgn{(i_1 i_2 \dotsc i_n )} = \ddsum{\sigma \in S_n}{}a_{1\sigma(1)}{\cdot}a_{2\sigma(2)}{\cdot}\dotsc{\cdot}a_{n\sigma(n)}{\cdot}\sgn{\sigma}
	$$
	эта формула называется \uwave{развернутой формулой для определителя}.
\end{defn}
\begin{rem}
	Запись определителя в терминах подстановок вместо перестановок обусловлена тем, что каждая подстановка может быть записана в стандартной двухрядной записи, а тогда в нижней строке будет фигурировать как раз перестановка из первой формулы:
	$$
		\sigma = \begin{pmatrix}
			1 & 2 & \dotsc & n \\
			i_1 & i_2 & \dotsc & i_n
		\end{pmatrix}
	$$
\end{rem}
\begin{rem}
	Это число ещё называют \uwave{определителем порядка} $n$ или \uwave{определителем} $n$-го \uwave{порядка}. Возможно обозначение $|A| = \det{A}$.
\end{rem}
По смыслу, в определителе суммирование ведется по перестановкам $\Rightarrow$ там $n!$ слагаемых (столько, сколько перестановок), причем эти слагаемые берутся со знаками $\Rightarrow$ ровно половина слагаемых с плюсом, половина - с минусом.

Каждое слагаемое это произведение элементов матрицы, которые берутся по одному из каждой строки и из каждого столбца. И каждое такое произведение берется со знаком, которое равно знаку подстановки, устанавливающей соответствие между номерами строк и номерами столбцов в этом произведении.

\textbf{Пример}: рассмотрим простейщий нетривиальный пример:
$$
	\begin{vmatrix}
		a_{11} & a_{12}\\
		a_{21} & a_{22}
	\end{vmatrix} = a_{11}{\cdot}a_{22}{\cdot}\sgn{(12)} + a_{12}{\cdot}a_{21}{\cdot}\sgn{(21)} = a_{11}{\cdot}a_{22} - a_{12 }{\cdot}a_{21}
$$
то есть произведение элементов по главной диагонали минус произведение элементов по побочной:
$$
	+ \begin{pmatrix}
		* & \cdot\\
		\cdot  & *
	\end{pmatrix} - 
	\begin{pmatrix}
		\cdot & * \\
		* & \cdot\\
	\end{pmatrix}
$$
\subsection*{Свойства определителя матрицы}
Удобно рассматривать определитель как функцию от набора строчек (столбцов) этой матрицы: 
$$\det{A} = \det{(A_1, \dotsc, A_n)}$$
где $A_1, \dotsc, A_n$ - строки матрицы $A$ (векторы из $\MR^n$).

\begin{enumerate}[label=\arabic*)]
	\item \textbf{Аддитивность}: при разложении $k$-ой строчки в сумму двух строк, определитель будет равен сумме определителей со слагаемыми $k$-ой строчки:
	$$
		\det{(A_1, \dotsc, A'_k + A''_k, \dotsc, A_n)} = \det{(A_1, \dotsc, A'_k , \dotsc, A_n)} + \det{(A_1, \dotsc,  A''_k, \dotsc, A_n)}
	$$ 
	\begin{proof}
		$$
			\det{A} = \ddsum{(i_1 \dotsc i_n)}{}a_{1{i_1}}{\cdot}\dotsc{\cdot}a_{k{i_k}}{\cdot}\dotsc{\cdot}a_{n{i_n}}{\cdot}\sgn{(i_1  \dotsc i_n )}, \, a_{k{i_k}} = a'_{k{i_k}} + a''_{k{i_k}} \Rightarrow \det{A} = 
		$$
		$$
			 = \ddsum{(i_1 \dotsc i_n)}{}a_{1{i_1}}{\cdot}\dotsc{\cdot}a'_{k{i_k}}{\cdot}\dotsc{\cdot}a_{n{i_n}}{\cdot}\sgn{(i_1  \dotsc i_n )} + \ddsum{(i_1 \dotsc i_n)}{}a_{1{i_1}}{\cdot}\dotsc{\cdot}a''_{k{i_k}}{\cdot}\dotsc{\cdot}a_{n{i_n}}{\cdot}\sgn{(i_1  \dotsc i_n )} = 
		$$
		$$
			=	\det{(A_1, \dotsc, A'_k , \dotsc, A_n)} + \det{(A_1, \dotsc,  A''_k, \dotsc, A_n)}
		$$
	\end{proof}
	\item \textbf{Однородность}: при умножении строчки на $\lambda$, исходный определитель умножиться на $\lambda$:
	$$
		\det{(A_1,\dotsc, \lambda{\cdot}A_k, \dotsc, A_n)} = \lambda {\cdot} \det{A}
	$$
	\begin{proof}
		$$
			\det{(A_1,\dotsc, \lambda{\cdot}A_k, \dotsc, A_n)} = \ddsum{(i_1 \dotsc i_n)}{}a_{1{i_1}}{\cdot}\dotsc{\cdot}(\lambda{\cdot}a_{k{i_k}}){\cdot}\dotsc{\cdot}a_{n{i_n}}{\cdot}\sgn{(i_1  \dotsc i_n )} =
		$$
		$$
			=	\lambda{\cdot}\ddsum{(i_1 \dotsc i_n)}{}a_{1{i_1}}{\cdot}\dotsc{\cdot}a_{k{i_k}}{\cdot}\dotsc{\cdot}a_{n{i_n}}{\cdot}\sgn{(i_1  \dotsc i_n )} = \lambda{\cdot}\det{A}
		$$
	\end{proof}
	\item \textbf{Линейность определителя по $k$-ой строке}: определитель - \textbf{полилинейная функция} от строк матрицы, то есть он линеен по каждой $k = 1,\dotsc,n$ строке;
	\begin{proof}
		Следует сразу из первых двух свойств.
	\end{proof}
	\item  \textbf{Кососимметричность}: при перестановке двух строк местами, определитель меняет знак:
	$$
		\det{(A_1,\dotsc, A_k,\dotsc,A_l,\dotsc,A_n)} = - \det{(A_1,\dotsc, A_l,\dotsc,A_k,\dotsc,A_n)}
	$$
	\begin{proof}
		$$
			A \xrightarrow[\text{ $k$-ой и $l$-ой строки}]{\text{транспозиция}} A', \,  \det{A} = \ddsum{\sigma \in S_n}{}a_{1\sigma(1)}{\cdot}\dotsc{\cdot}a_{k\sigma(k)}{\cdot}\dotsc{\cdot}a_{l\sigma(l)}{\cdot}\dotsc{\cdot}a_{n\sigma(n)}{\cdot}\sgn{\sigma} \Rightarrow
		$$
		$$
			\Rightarrow \det{A} = \ddsum{\sigma \in S_n}{}a_{1\sigma(1)}{\cdot}\dotsc{\cdot}a_{l\sigma(l)}{\cdot}\dotsc{\cdot}a_{k\sigma(k)}{\cdot}\dotsc{\cdot}a_{n\sigma(n)}{\cdot}\sgn{\sigma} 
		$$
		$$
			\pi = \sigma{\cdot}\tau, \, \tau = (kl) \Rightarrow \pi{\cdot}\tau = \sigma{\cdot}\tau{\cdot}\tau = \sigma, \, \sgn{\sigma} = \sgn{\tau}{\cdot}\sgn{\pi} = - \sgn{\pi}
		$$
		Заметим, что $\pi$ пробегает всё $S_n$ по одному разу, если $\sigma = \pi{\cdot}\tau$ пробегает всё $S_n$ по одному разу, поскольку $\sigma$ можно однозначно восстановить по $\pi$: число разных $\sigma =$ числу разных $\pi = n!$, а если для разных $\sigma$ две $\pi$ совпали, то и $\sigma$ должны совпасть $\Rightarrow$ инъективность и конечность множества дают биективность. Следовательно, можно суммировать по $\pi$:
		$$
			\ddsum{\sigma \in S_n}{}a_{1\sigma(1)}{\cdot}\dotsc{\cdot}a_{l\sigma(l)}{\cdot}\dotsc{\cdot}a_{k\sigma(k)}{\cdot}\dotsc{\cdot}a_{n\sigma(n)}{\cdot}\sgn{\sigma}  = \ddsum{\pi \in S_n}{}a'_{1\pi(1)}{\cdot}\dotsc{\cdot}a'_{k\pi(k)}{\cdot}\dotsc{\cdot}a'_{l\pi(l)}{\cdot}\dotsc{\cdot}a'_{n\pi(n)}{\cdot}\sgn{\sigma} =
		$$
		$$
			=	\ddsum{\pi \in S_n}{}a'_{1\pi(1)}{\cdot}\dotsc{\cdot}a'_{k\pi(k)}{\cdot}\dotsc{\cdot}a'_{l\pi(l)}{\cdot}\dotsc{\cdot}a'_{n\pi(n)}{\cdot}(-\sgn{\pi}) = -\det{A'}
		$$
	\end{proof}
	\item $\exists \, k \colon A_k = 0 \Rightarrow \det{A} = 0$;
	\begin{proof}
		$$
			\det{(A_1,\dotsc, A_k, \dotsc ,A_n)} = \det{(A_1,\dotsc, 0{\cdot}A_k, \dotsc, A_n)} = 0 {\cdot}\det{(A_1,\dotsc, A_k, \dotsc ,A_n)}=0
		$$
	\end{proof}
	\item $\exists \, k \neq l \colon A_k = A_l \Rightarrow \det{A} = 0$;
	\begin{proof}
		$$
			\det{A} = \det{(A_1,\dotsc, A_k,\dotsc,A_l, \dotsc,A_n)} = -\det{(A_1,\dotsc, A_l,\dotsc,A_k, \dotsc,A_n)} = |A_k = A_l| =
		$$
		$$
			= - \det{(A_1,\dotsc, A_k,\dotsc,A_l, \dotsc,A_n)} = -\det{A} \Rightarrow \det{A} = 0
		$$
	\end{proof}
	\item $\exists \, k \neq l \colon A_l = \lambda{\cdot}A_k \Rightarrow \det{A} = 0$;
	\begin{proof}
		$$
			\det{A} = \det{(A_1,\dotsc, A_k,\dotsc,A_l, \dotsc,A_n)} = \det{(A_1,\dotsc, A_k,\dotsc,\lambda{\cdot}A_k, \dotsc,A_n)} =
		$$
		$$
			= \lambda{\cdot}\det{(A_1,\dotsc, A_k,\dotsc,A_k, \dotsc,A_n)} \Rightarrow \det{A} = 0
		$$
	\end{proof}
	\item При ЭП строк $1$-го типа (прибавление к одной строке другой, умноженной на скаляр), определитель не меняется:
	$$
		A \xrightarrow{\text{ЭП}1} A' \Rightarrow \det{A} = \det{A'}
	$$
	\begin{proof}
		$$
			\det{A'} = \det{(A_1,\dotsc, A_k + \lambda{\cdot}A_l,\dotsc,A_l, \dotsc,A_n)} = 
		$$
		$$
			=	\det{(A_1,\dotsc, A_k ,\dotsc,A_l, \dotsc,A_n)} + \lambda{\cdot}\det{(A_1,\dotsc, A_l ,\dotsc,A_l, \dotsc,A_n)} = \det{A} + 0= \det{A}
		$$
	\end{proof}
	\item При транспонировании определитель матрицы не меняется:
	$$
		\det{A} = \det{A^T}
	$$
	\begin{proof}
		$$
			\det{A^T} = \ddsum{\sigma \in S_n}{}a^T_{1\sigma(1)}{\cdot}\dotsc{\cdot}a^T _{n\sigma(n)}{\cdot}\sgn{\sigma} = \ddsum{\sigma \in S_n}{}a_{\sigma(1)1}{\cdot}\dotsc{\cdot}a _{\sigma(n)n}{\cdot}\sgn{\sigma}
		$$
		Переставим множители в этом произведении. Посмотрим, какая будет перестановка:
		$$
			\sigma(j) = i \Rightarrow j = \sigma^{-1}(i )
		$$
		где $i$ - это номер строки, а $j$ - это номер столбца. Тогда:
		$$
			\ddsum{\sigma \in S_n}{}a_{\sigma(1)1}{\cdot}\dotsc{\cdot}a _{\sigma(n)n}{\cdot}\sgn{\sigma} = \ddsum{\sigma \in S_n}{}a_{1\sigma^{-1}(1)}{\cdot}\dotsc{\cdot}a _{n\sigma^{-1}(n)}{\cdot}\sgn{\sigma}
		$$
		Заметим, что если $\sigma$ пробегает все подстановки по одному разу, то $\pi =\sigma^{-1}$ тоже пробегает множество всех подстановок по одному разу: если две подстановки разные, то и обратные подстановки у них будут разные. Можем переписать нашу сумму:
		$$
			\ddsum{\sigma \in S_n}{}a_{1\sigma^{-1}(1)}{\cdot}\dotsc{\cdot}a _{n\sigma^{-1}(n)}{\cdot}\sgn{\sigma} = \ddsum{\pi \in S_n}{}a_{1\pi(1)}{\cdot}\dotsc{\cdot}a _{n\pi(n)}{\cdot}\sgn{(\pi^{-1})} = 
		$$
		$$
			= \ddsum{\pi \in S_n}{}a_{1\pi(1)}{\cdot}\dotsc{\cdot}a _{n\pi(n)}{\cdot}\sgn{\pi} = \det{A}
		$$
	\end{proof}
	\item Все свойства выше будут верны для определителя, как функции от столбцов матрицы;
	\begin{proof}
		При транспонировании строки и столбцы меняются ролями, а поскольку определитель транспонированной матрицы такой же, то свойства будут верными.
	\end{proof}
\end{enumerate}

Мы определили определитель с помощью развернутой формулы, но на самом деле она не очень удобна для его вычисления, поскольку в ней очень много слагаемых и оно очень растёт с ростом $n$. Нужен более эффективный метод вычисления определителей.

\subsection*{Метод вычисления $\det$ привидением матрицы к треугольному виду}

Пусть у нас есть матрица $A$ и ЭП строк мы её приводим к ступенчатому виду $A^*$. Поскольку матрица квадратная, то ниже диагонали у неё всегда должны быть нули и она будет иметь вид:
$$
	A \xrightarrow{\text{ЭП строк}} A^* = 
	\begin{pmatrix}
		\lambda_1 & * & \dotsc & * \\
		0 & \lambda_2 &  \dotsc & *\\
		\vdots & \vdots & \ddots & \vdots\\
		0 & 0 & \dotsc & \lambda_n
	\end{pmatrix}
$$
Данный вид называется \uwave{треугольным}. Любую квадратную матрицу мы можем сделать треугольной, тогда определитель новой матрицы запишется так:
$$
	\det{A^*} = \det{A}{\cdot}(-1)^p{\cdot}\mu_1{\cdot}\dotsc{\cdot}\mu_q
$$
где $p$ - количество ЭП$2$, которые мы совершали перед приведением к $A^*$, а когда мы делаем ЭП$3$, то определитель тоже умножается на коэффициенты этих ЭП: $\mu_1, \dotsc,\mu_q$. При ЭП$1$ определитель не меняется, как мы уже знаем.
\begin{prop}
	$$
		\begin{vmatrix}
			\lambda_1 & * & \dotsc & * \\
			0 & \lambda_2 &  \dotsc & *\\
			\vdots & \vdots & \ddots & \vdots\\
			0 & 0 & \dotsc & \lambda_n
		\end{vmatrix} = \lambda_1{\cdot}\dotsc{\cdot}\lambda_n
	$$
\end{prop}
\begin{proof}
	Воспользуемся однородностью и рассмотрим последнюю строку матрицы:
	$$
		\begin{vmatrix}
			\lambda_1 & * & \dotsc & * \\
			0 & \lambda_2 &  \dotsc & *\\
			\vdots & \vdots & \ddots & \vdots\\
			0 & 0 & \dotsc & \lambda_n
		\end{vmatrix} = 
		\lambda_n{\cdot}
		\begin{vmatrix}
			\lambda_1 & * & \dotsc & * \\
			0 & \lambda_2 &  \dotsc & *\\
			\vdots & \vdots & \ddots & \vdots\\
			0 & 0 & \dotsc & 1
		\end{vmatrix}
	$$
	Применим ЭП$1$ и будем вычитать последнюю строку из остальных строк с подходящими коэффициентами, чтобы обнулить все элементы выше $1$ в столбце, тогда:
	$$
		\lambda_n{\cdot}
		\begin{vmatrix}
			\lambda_1 & * & \dotsc & * & * \\
			0 & \lambda_2 &  \dotsc & * & *\\
			\vdots & \vdots & \ddots & \vdots & \vdots\\
			0 & 0 & \dotsc & \lambda_{n-1} & *\\
			0 & 0 & \dotsc & 0 & 1
		\end{vmatrix} = 
		\lambda_n{\cdot}
		\begin{vmatrix}
			\lambda_1 & * & \dotsc & * &  0 \\
			0 & \lambda_2 &  \dotsc & * & 0\\
			\vdots & \vdots & \ddots & \vdots & \vdots\\
			0 & 0 & \dotsc & \lambda_{n-1} & 0\\
			0 & 0 & \dotsc & 0 & 1
		\end{vmatrix}
	$$
	Аналогично проделываем операции с $\lambda_{n-1}$:
	$$
		\lambda_n{\cdot}
		\begin{vmatrix}
			\lambda_1 & * & \dotsc & * &  0 \\
			0 & \lambda_2 &  \dotsc & * & 0\\
			\vdots & \vdots & \ddots & \vdots & \vdots\\
			0 & 0 & \dotsc & \lambda_{n-1} & 0\\
			0 & 0 & \dotsc & 0 & 1
		\end{vmatrix} = 
		\lambda_n{\cdot}\lambda_{n-1}{\cdot}
		\begin{vmatrix}
			\lambda_1 & * & \dotsc & * &  0 \\
			0 & \lambda_2 &  \dotsc & * & 0\\
			\vdots & \vdots & \ddots & \vdots & \vdots\\
			0 & 0 & \dotsc & 1 & 0\\
			0 & 0 & \dotsc & 0 & 1
		\end{vmatrix}= 
		\lambda_n{\cdot}\lambda_{n-1}{\cdot}
		\begin{vmatrix}
			\lambda_1 & * & \dotsc & 0 &  0 \\
			0 & \lambda_2 &  \dotsc & 0 & 0\\
			\vdots & \vdots & \ddots & \vdots & \vdots\\
			0 & 0 & \dotsc & 1 & 0\\
			0 & 0 & \dotsc & 0 & 1
		\end{vmatrix} = \dotsc 
	$$
	И так далее, в итоге мы получим:
	$$
		\dotsc = \lambda_1{\cdot}\lambda_2{\cdot}\dotsc{\cdot} \lambda_n{\cdot}
		\begin{vmatrix}
			1 & 0 & \dotsc & 0 &  0 \\
			0 & 1 &  \dotsc & 0 & 0\\
			\vdots & \vdots & \ddots & \vdots & \vdots\\
			0 & 0 & \dotsc & 1 & 0\\
			0 & 0 & \dotsc & 0 & 1
		\end{vmatrix} = \lambda_1{\cdot}\lambda_2{\cdot}\dotsc{\cdot} \lambda_n{\cdot}1{\cdot}\dotsc{\cdot}1{\cdot}\sgn{\VE} = \lambda_1{\cdot}\lambda_2{\cdot}\dotsc{\cdot} \lambda_n{\cdot}
	$$
\end{proof}
Собирая всё в одном месте, после приведения нашей матрицы к треугольной, мы найдем определитель исходной матрицы $A$:
$$
	\det{A} = \det{A^*}{\cdot}(-1)^p{\cdot}\dfrac{1}{\mu_1}{\cdot}\dotsc{\cdot}\dfrac{1}{\mu_q}= \lambda_1{\cdot}\dotsc{\cdot}\lambda_n{\cdot}(-1)^p{\cdot}\dfrac{1}{\mu_1}{\cdot}\dotsc{\cdot}\dfrac{1}{\mu_q}
$$

\end{document}