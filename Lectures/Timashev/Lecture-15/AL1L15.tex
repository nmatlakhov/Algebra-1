\documentclass[12pt]{article}
\usepackage[left=1cm, right=1cm, top=2cm,bottom=1.5cm]{geometry} 

\usepackage[parfill]{parskip}
\usepackage[utf8]{inputenc}
\usepackage[T2A]{fontenc}
\usepackage[russian]{babel}
\usepackage{enumitem}
\usepackage[normalem]{ulem}
\usepackage{amsfonts, amsmath, amsthm, amssymb, mathtools,xcolor}
\usepackage{blkarray}

\usepackage{tabularx}
\usepackage{hhline}

\usepackage{accents}
\usepackage{fancyhdr}
\pagestyle{fancy}
\renewcommand{\headrulewidth}{1.5pt}
\renewcommand{\footrulewidth}{1pt}

\usepackage{graphicx}
\usepackage[figurename=Рис.]{caption}
\usepackage{subcaption}
\usepackage{float}

%%Наименование папки откуда забирать изображения
\graphicspath{ {./images/} }

%%Изменение формата для ввода доказательства
\renewcommand{\proofname}{$\square$  \nopunct}
\renewcommand\qedsymbol{$\blacksquare$}

%%Изменение отступа на таблицах
\addto\captionsrussian{%
	\renewcommand{\proofname}{$\square$ \nopunct}%
}
%% Римские цифры
\newcommand{\RN}[1]{%
	\textup{\uppercase\expandafter{\romannumeral#1}}%
}

%% Для удобства записи
\newcommand{\MR}{\mathbb{R}}
\newcommand{\MC}{\mathbb{C}}
\newcommand{\MQ}{\mathbb{Q}}
\newcommand{\MN}{\mathbb{N}}
\newcommand{\MZ}{\mathbb{Z}}
\newcommand{\MTB}{\mathbb{T}}
\newcommand{\MTI}{\mathbb{I}}
\newcommand{\MI}{\mathrm{I}}
\newcommand{\MCI}{\mathcal{I}}
\newcommand{\MJ}{\mathrm{J}}
\newcommand{\MH}{\mathrm{H}}
\newcommand{\MT}{\mathrm{T}}
\newcommand{\MU}{\mathcal{U}}
\newcommand{\MV}{\mathcal{V}}
\newcommand{\MB}{\mathcal{B}}
\newcommand{\MF}{\mathcal{F}}
\newcommand{\MW}{\mathcal{W}}
\newcommand{\ML}{\mathcal{L}}
\newcommand{\MP}{\mathcal{P}}
\newcommand{\VN}{\varnothing}
\newcommand{\VE}{\varepsilon}
\newcommand{\dx}{\, dx}
\newcommand{\dy}{\, dy}
\newcommand{\dz}{\, dz}
\newcommand{\dd}{\, d}


\theoremstyle{definition}
\newtheorem{defn}{Опр:}
\newtheorem{rem}{Rm:}
\newtheorem{prop}{Утв.}
\newtheorem{exrc}{Упр.}
\newtheorem{problem}{Задача}
\newtheorem{lemma}{Лемма}
\newtheorem{theorem}{Теорема}
\newtheorem{corollary}{Следствие}

\newenvironment{cusdefn}[1]
{\renewcommand\thedefn{#1}\defn}
{\enddefn}

\DeclareRobustCommand{\divby}{%
	\mathrel{\text{\vbox{\baselineskip.65ex\lineskiplimit0pt\hbox{.}\hbox{.}\hbox{.}}}}%
}
\DeclareRobustCommand{\ndivby}{\mkern-1mu\not\mathrel{\mkern4.5mu\divby}\mkern1mu}


%Короткий минус
\DeclareMathSymbol{\SMN}{\mathbin}{AMSa}{"39}
%Длинная шапка
\newcommand{\overbar}[1]{\mkern 1.5mu\overline{\mkern-1.5mu#1\mkern-1.5mu}\mkern 1.5mu}
%Функция знака
\DeclareMathOperator{\sgn}{sgn}

%Функция ранга
\DeclareMathOperator{\rk}{\text{rk}}
\DeclareMathOperator{\diam}{\text{diam}}


%Обозначение константы
\DeclareMathOperator{\const}{\text{const}}

\DeclareMathOperator{\codim}{\text{codim}}

\DeclareMathOperator*{\dsum}{\displaystyle\sum}
\newcommand{\ddsum}[2]{\displaystyle\sum\limits_{#1}^{#2}}

%Интеграл в большом формате
\DeclareMathOperator{\dint}{\displaystyle\int}
\newcommand{\ddint}[2]{\displaystyle\int\limits_{#1}^{#2}}
\newcommand{\ssum}[1]{\displaystyle \sum\limits_{n=1}^{\infty}{#1}_n}

\newcommand{\smallerrel}[1]{\mathrel{\mathpalette\smallerrelaux{#1}}}
\newcommand{\smallerrelaux}[2]{\raisebox{.1ex}{\scalebox{.75}{$#1#2$}}}

\newcommand{\smallin}{\smallerrel{\in}}
\newcommand{\smallnotin}{\smallerrel{\notin}}

\newcommand*{\medcap}{\mathbin{\scalebox{1.25}{\ensuremath{\cap}}}}%
\newcommand*{\medcup}{\mathbin{\scalebox{1.25}{\ensuremath{\cup}}}}%

\makeatletter
\newcommand{\vast}{\bBigg@{3.5}}
\newcommand{\Vast}{\bBigg@{5}}
\makeatother

%Промежуточное значение для sup\inf, поскольку они имеют разную высоту
\newcommand{\newsup}{\mathop{\smash{\mathrm{sup}}}}
\newcommand{\newinf}{\mathop{\mathrm{inf}\vphantom{\mathrm{sup}}}}

%Скалярное произведение
\newcommand{\inner}[2]{\left\langle #1, #2 \right\rangle }
\newcommand{\linsp}[1]{\left\langle #1 \right\rangle }
\newcommand{\linmer}[2]{\left\langle #1 \vert #2\right\rangle }

%Подпись символов снизу
\newcommand{\ubar}[1]{\underaccent{\bar}{#1}}

%% Шапка для букв сверху
\newcommand{\wte}[1]{\widetilde{#1}}
\newcommand{\wht}[1]{\widehat{#1}}
\newcommand{\ovl}[1]{\overline{#1}}

%%Трансформация Фурье
\newcommand{\fourt}[1]{\mathcal{F}\left(#1\right)}
\newcommand{\ifourt}[1]{\mathcal{F}^{-1}\left(#1\right)}

%%Символ вектора
\newcommand{\vecm}[1]{\overrightarrow{#1\,}}

%%Пространстов матриц
\newcommand{\matsq}[1]{\operatorname{Mat}_{#1}}
\newcommand{\mat}[2]{\operatorname{Mat}_{#1, #2}}

%Оператор для действ и мнимых чисел
\DeclareMathOperator{\IM}{\operatorname{Im}}
\DeclareMathOperator{\RE}{\operatorname{Re}}
\DeclareMathOperator{\li}{\operatorname{li}}
\DeclareMathOperator{\GL}{\operatorname{GL}}
\DeclareMathOperator{\SL}{\operatorname{SL}}
\DeclareMathOperator{\Char}{\operatorname{char}}
\DeclareMathOperator\Arg{Arg}

%Делимость чисел
\newcommand{\modn}[3]{#1 \equiv #2 \; (\bmod \; #3)}


%%Взятие в скобки, модули и норму
\newcommand{\parfit}[1]{\left( #1 \right)}
\newcommand{\modfit}[1]{\left| #1 \right|}
\newcommand{\sqparfit}[1]{\left\{ #1 \right\}}
\newcommand{\normfit}[1]{\left\| #1 \right\|}

%%Функция для обозначения равномерной сходимости по множеству
\newcommand{\uconv}[1]{\overset{#1}{\rightrightarrows}}
\newcommand{\uconvm}[2]{\overset{#1}{\underset{#2}{\rightrightarrows}}}


%%Функция для обозначения нижнего и верхнего интегралов
\def\upint{\mathchoice%
	{\mkern13mu\overline{\vphantom{\intop}\mkern7mu}\mkern-20mu}%
	{\mkern7mu\overline{\vphantom{\intop}\mkern7mu}\mkern-14mu}%
	{\mkern7mu\overline{\vphantom{\intop}\mkern7mu}\mkern-14mu}%
	{\mkern7mu\overline{\vphantom{\intop}\mkern7mu}\mkern-14mu}%
	\int}
\def\lowint{\mkern3mu\underline{\vphantom{\intop}\mkern7mu}\mkern-10mu\int}

%%След матрицы
\DeclareMathOperator*{\tr}{tr}

\makeatletter
\renewcommand*\env@matrix[1][*\c@MaxMatrixCols c]{%
	\hskip -\arraycolsep
	\let\@ifnextchar\new@ifnextchar
	\array{#1}}
\makeatother


%% Переопределение функции хи, чтобы выглядела более приятно
\makeatletter
\@ifdefinable\@latex@chi{\let\@latex@chi\chi}
\renewcommand*\chi{{\@latex@chi\smash[t]{\mathstrut}}} % want only bottom half of \mathstrut
\makeatletter

\setcounter{MaxMatrixCols}{20}

\begin{document}
\lhead{Алгебра-\RN{1}}
\chead{Тимашев Д.А.}
\rhead{Лекция - 15}

\section*{Многочлены}
\begin{defn}
	Пусть $K$ - кольцо (коммутативное, ассоциативное, с единицей). \uwave{Кольцо многочленов} от одной переменной над кольцом $K$ это кольцо $K[x]$, удовлетворяющее следующим условиям:
	\begin{enumerate}[label=\arabic*)]
		\item $K \subset K[x]$ или $K[x]$ содержит подкольцо, изоморфное $K$;
		\item $x \in K[x] \colon x \not\in K$, где выделенный элемент $x$ называется \uwave{переменной};
		\item $\forall f \in K[x], \, \exists!$  представление $f = \ddsum{k =0 }{\infty}a_k{\cdot}x^k, \,  a_k \in K, \, \exists \, n \colon \forall k > n, \, a_k = 0$;
	\end{enumerate}
\end{defn}

\begin{theorem}
	Кольцо многочленов от одной переменной $K[x]$  единственно с точностью до изоморфизма.
\end{theorem}
\begin{proof}\hfill
	\begin{enumerate}[label=\arabic*)]
		\item Пусть $f = \ddsum{n = 0}{\infty}a_nx^n, \, g = \ddsum{n = 0}{\infty}b_nx^n \in K[x]$, тогда:
		$$
			f + g = \ddsum{n = 0}{\infty}(a_n + b_n){\cdot}x^n
		$$
		$$
			f{\cdot}g = \ddsum{k = 0}{\infty}a_k{\cdot}x^k{\cdot}\ddsum{l = 0}{\infty}b_l{\cdot}x^l = \ddsum{k,l \geq 0}{}a_k{\cdot}b_l{\cdot}x^{k+l} = \ddsum{n = 0}{\infty}c_n{\cdot}x^n, \, \forall n \geq 0, \, c_n = \ddsum{\substack{k,l \geq 0\\k + l = n}}{}a_k{\cdot}b_l
		$$
		\textbf{\uline{Вывод}}: коэффициенты суммы и произведения многочленов зависят только от коэффициентов исходных многочленов;
		\item Пользуясь $1)$ мы можем построить изоморфизм между любыми двумя кольцами многочленов. Пусть $K[y]$ - другое кольцо многочленов. Построим изоморфизм $\varphi \colon K[x]\xrightarrow[\sim]{} K[y]$: 
		$$
			\forall f = a_0 + a_1 x + \dotsc + a_n x^n, \, \varphi(f) = a_0 + a_1y + a_2y^2 + \dotsc + a_ny^n
		$$
		\textbf{\uline{Биективность}}: вытекает из того, что каждый многчлен единственным образом представляется в виде линейной комбинации одночленов $\Rightarrow$ взаимнооднозначно соответствует последовательности коэффициентов $\Rightarrow$ два многочлена имеют одну и ту же последовательность коэффициентов $\Rightarrow$ соответствие взаимно однозначно.
		
		\textbf{\uline{Согласованность с операциями}}: вытекат из того, что результат операции над двумя многочленами определяется только тем, какие были коэффициенты у исходных многочленов:
		$$
			\varphi(f + g) = \ddsum{n = 0}{\infty}(a_n + b_n){\cdot}y^n = \ddsum{n = 0}{\infty}a_n{\cdot}y^n + \ddsum{n = 0}{\infty}b_n{\cdot}y^n = \varphi(f) + \varphi(g)
		$$
		$$
			\varphi(f{\cdot}g) = \ddsum{n = 0}{\infty}c_n{\cdot}y^n = \ddsum{k = 0}{\infty}a_k{\cdot}y^k{\cdot}\ddsum{l = 0}{\infty}b_l{\cdot}y^l  = \varphi(f){\cdot}\varphi(g), \, \forall n \geq 0, \, c_n = \ddsum{\substack{k,l \geq 0\\k + l = n}}{}a_k{\cdot}b_l
		$$
		Таким образом, любые два кольца многочленов от одной переменной над кольцом коэффициентов $K$ изоморфны друг другу;
	\end{enumerate}
\end{proof}

\newpage
\section*{Существование кольца многочленов}
Построим модель кольца, которое удовлетворяет определению кольца многочленов.

\textbf{\uline{Модель кольца многочленов}}: Многочлен задается последовательностью своих коэффициентов $\Rightarrow$ можно считать, что эта последовательность бесконечна, но просто продолжена $0$ до бесконечности, но реально в ней лишь конечное число ненулевых членов. 
\begin{defn}
	Последовательность $a = (a_0, a_1,a_2,\dotsc, a_k,\dotsc)$ элементов $a_k \in K$ называется \uwave{финитной}, если начиная с некоторого места, её члены равны нулю:
	$$
		\exists \, n \in \MN \colon \forall k > n, \, a_k = 0
	$$
\end{defn}
\textbf{\uline{Обозначение}}: Пусть $K^{\infty}$ - множество финитных последовательностей элементов кольца $K$. 

Хотим превратить множество $K^{\infty}$ в кольцо, которое бы удовлетворяло определению кольца многочленов. Введем в нём операции, которые происходят над последовательностями коэффициентов в кольце многочленов, если бы это кольцо существовало.

\textbf{\uline{Операции}}:
\begin{enumerate}
	\item[($+$):] $\forall a, b \in K^{\infty}, \, a+ b = (a_0 + b_0, a_1 + b_1, \dotsc, a_k + b_k, \dotsc )$;
	\item[($\, \cdot\, $):] $\forall a, b \in K^{\infty}, \, a{\cdot}b = \left(a_0{\cdot}b_0, a_0{\cdot}b_1 + a_1{\cdot}b_0, a_0{\cdot}b_2 + a_1{\cdot}b_1 + a_2{\cdot}b_0, \dotsc, a_0{\cdot}b_n + a_1{\cdot}b_{n-1} + \dotsc + a_n{\cdot} b_0, \dotsc \right)$, где:
	$$
		(a{\cdot}b)_n = a_0 {\cdot}b_n + a_1{\cdot}b_{n-1} + \dotsc + a_n{\cdot} b_0 = \ddsum{\substack{k,l \geq 0\\k + l = n}}{}a_k{\cdot}b_l
	$$
\end{enumerate}

Заметим, что при сложении сумма будет финитна, поскольку обе последовательности финитны, а сложение идёт покомпонентно. При умножении сумма также будет финитна, поскольку либо $k$ будет достаточно большое, либо $l$ так, чтобы $a_kb_l = 0$, в силу финитностни исходных последовательностей.
\begin{prop}
	Множество $K^{\infty}$ замкнуто относительно операций сложения и умножения, определенных выше.
\end{prop}
\begin{proof}
	Рассмотрим операции в явном виде:
	$$
		\forall a,b \in K^{\infty}, \, \exists \, n,m \in \MN \colon \forall k > n, \, a_k = 0, \, \forall l > m, \, b_l = 0 \Rightarrow \forall p > r = \max(n,m), \, a_p = 0, b_p = 0 \Rightarrow
	$$
	$$
		\Rightarrow a + b = (a_0 + b_0, a_1 + b_1, \dotsc, a_r + b_r, 0 , \dotsc ) \Rightarrow a + b \in K^{\infty}
	$$
	$$
		\Rightarrow a{\cdot}b = (a_0{\cdot}b_0,  \dotsc, a_0{\cdot}b_r + a_1{\cdot} b_{r-1} + \dotsc + a_r {\cdot}b_0, \dotsc, a_0{\cdot}b_{2p} + a_1{\cdot}b_{2p -1} + \dotsc + a_p{\cdot}b_p + \dotsc + a_{2p}{\cdot}b_0, \dotsc)
	$$
	$$
		a_0{\cdot}b_{2p} +  \dotsc + a_p{\cdot}b_p + \dotsc + a_{2p}{\cdot}b_0 = a_0{\cdot}0  + \dotsc + 0{\cdot}0 + \dotsc + 0{\cdot}b_0 = 0 \Rightarrow a{\cdot}b \in K^{\infty}
	$$
\end{proof}

\textbf{\uline{Проверка аксиом кольца}}:
\begin{enumerate}[label=\arabic*)]
	\item \textbf{Коммутативность сложения}: следует из коммутативности операции в кольце $K$, $\forall a,b\in K^{\infty}$:
	$$
		a + b  = (a_0 + b_0, a_1 + b_1, \dotsc, a_n + b_n, \dotsc) = (b_0 + a_0, b_1 + a_1, \dotsc, b_n + a_n, \dotsc) = b + a
	$$
	\item \textbf{Коммутативность умножения}: следует из коммутативности операции в кольце $K$, $\forall a,b\in K^{\infty}$:
	$$
		a{\cdot}b = (a_0{\cdot}b_0, \dotsc, a_0{\cdot}b_n + a_1{\cdot}b_{n-1} + \dotsc + a_n{\cdot}b_0, \dotsc ) = (b_0{\cdot}a_0, \dotsc, b_0{\cdot}a_n + b_1{\cdot}a_{n-1} + \dotsc + b_n {\cdot}a_0, \dotsc ) = b{\cdot}a
	$$
	\item \textbf{Ассоциативность сложения}: следует из ассоциативности операции в кольце $K$, $\forall a,b,c\in K^{\infty}$:
	$$
		a + (b + c) = (a_0 + (b_0 + c_0), \dotsc, a_n + (b_n + c_n), \dotsc ) = ((a_0 + b_0) + c_0, \dotsc, (a_n + b_n) + c_n, \dotsc) = (a + b) + c
	$$
	\item \textbf{Ассоциативность умножения}: $\forall a,b,c \in K^{\infty}$:
	$$
		a{\cdot}b = f,\, (a{\cdot}b){\cdot}c = g, \, b{\cdot}c = u, \, a{\cdot}(b{\cdot}c) =v 
	$$
	$$
		\forall n \geq 0, \, g_n = \ddsum{\substack{k,l \geq 0\\k + l = n}}{}f_k{\cdot}c_l = \ddsum{\substack{k,l \geq 0\\k + l = n}}{}\bigg(\ddsum{\substack{i,j \geq 0\\i + j = k}}{}a_i{\cdot} b_j \bigg){\cdot}c_l = \ddsum{\substack{i,j,l \geq 0\\i + j + l = n}}{}a_i{\cdot}b_j{\cdot}c_l
	$$
	$$
		\forall n \geq 0, \, v_n = \ddsum{\substack{i,m \geq 0\\i + m = n}}{}a_i{\cdot}u_m = \ddsum{\substack{i,m \geq 0\\i + m = n}}{}a_i{\cdot}\bigg(\ddsum{\substack{j,l \geq 0\\j + l = m}}{}b_j{\cdot}c_l\bigg) = \ddsum{\substack{i,j,l \geq 0\\i + j + l = n}}{}a_i{\cdot}b_j{\cdot}c_l
	$$
	Следовательно, $g = v \Rightarrow$ ассоциативность умножения выполнена;
	\item \textbf{Дистрибутивность}: $\forall a,b,c \in K^{\infty}$:
	$$
		a{\cdot}(b + c) = (a_0{\cdot}(b_0 + c_0), \dotsc, a_0{\cdot}(b_n + c_n) + a_1{\cdot}(b_{n-1} + c_{n-1}) + \dotsc + a_n{\cdot}(b_0 + c_0), \dotsc ) =
	$$
	$$
		=(a_0{\cdot}b_0 + a_0{\cdot}c_0, \dotsc, a_0{\cdot}b_n + a_1{\cdot}b_{n-1} +\dotsc + a_n{\cdot}b_0 +  a_0{\cdot}c_n + a_1{\cdot}c_{n-1} +\dotsc + a_n{\cdot}c_0 , \dotsc) = a{\cdot}b + a{\cdot}c
	$$
	\item \textbf{Существование нулевого элемента}: $\forall a \in K^{\infty},\, \exists \, 0$:
	$$
		 0 = (0, 0, \dotsc, 0,\dotsc) \colon a + 0 = (a_0 + 0, \dotsc, a_j + 0, \dotsc) = (a_0,\dotsc, a_j,\dotsc) = a
	$$
	\item \textbf{Существование противоположного элемента}: $\forall a \in K^{\infty},\, \exists \, -a \in K^{\infty}$:
	$$
		-a = (-a_0, -a_1, \dotsc, -a_j, \dotsc), \,  a +(-a) = (a_0 - a_0, a_1 - a_1,\dotsc, a_j - a_j, \dotsc) = (0,0,\dotsc, 0) = 0
	$$
	\item \textbf{Существование единичного элемента}: $\forall a \in K^{\infty}, \, \exists \, 1$:
	$$
		1 = (1, 0,\dotsc, 0, \dotsc), \, a{\cdot}1 = (a_0{\cdot}1, a_0{\cdot}0 + a_1{\cdot}0, 0 + 0 + a_2, \dotsc, 0 + 0+ \dotsc + a_n{\cdot}1, \dotsc) = a
	$$
\end{enumerate}
Таким образом, множество финитных последовательностей $K^{\infty}$ является коммутативным, ассоциативным кольцом с единицей.

\textbf{\uline{Проверка свойств кольца многочленов}}:
\begin{enumerate}[label=\arabic*)]
	\item $K^{\infty} \supset \{a = (a_0, 0, 0, \dotsc, 0, \dotsc) \colon a_0 \in K\}$, поймем как они складываются и умножаются:
	$$
		(a_0, 0, \dotsc, 0, \dotsc ) + (b_0, 0, \dotsc, 0, \dotsc ) = (a_0 + b_0, 0,\dotsc, 0, \dotsc )
	$$
	$$
		(a_0, 0, \dotsc, 0, \dotsc ){\cdot}(b_0, 0, \dotsc, 0, \dotsc ) = (a_0{\cdot}b_0, 0, \dotsc, 0,\dotsc)
	$$
	Видим, что это множество замкнуто относительно сложения и умножения. Очевидно, оно непусто, противоположные элементы есть из свойств кольца $\Rightarrow$ это подкольцо. При этом это подкольцо будет изоморфно кольцу $K$, его элементы можно отождествить с элементами $K$:
	$$
		\varphi\colon \{a = (a_0, 0, 0, \dotsc, 0, \dotsc) \colon a_0 \in K\} \xrightarrow[\sim]{} K 
	$$
	$$
		\forall x = (x_0, 0, 0, \dotsc, 0, \dotsc) \in \{a = (a_0, 0, 0, \dotsc, 0, \dotsc) \colon a_0 \in K\}, \, \varphi(x) = x_0
	$$
	Следовательно, $K^{\infty}$ содержит \textbf{подкольцо}, изоморфное кольцу $K$. Ради простоты, можем считать, что $K$ содержится в $K^{\infty}$;
	\item \textbf{Переменная}: $x = (0,1,0, \dotsc, 0, \dotsc) \in K^{\infty}$ и $x \not\in \{a = (a_0, 0, 0, \dotsc, 0, \dotsc) \colon a_0 \in K\}\simeq K$;
	\newpage
	\item \textbf{Одночлены}: $x^n = (0, 0,\dotsc, 0, 1,0,\dotsc, 0, \dotsc)$, где $x^n_n = 1, \, \forall m \neq n, \, x^n_m = 0$. Докажем это:
	
	\textbf{\uline{База}}: $n = 0$:
	$$
		x^0 = 1 \Rightarrow x^0 = (1,0,\dotsc, 0, \dotsc)
	$$
	\textbf{\uline{Шаг индукции}}: пусть верно для $n$, рассмотрим $n+1$:
	$$
		x^{n+1}	= x^n{\cdot}x = (0, 0, \dotsc, 0, x^n_n = 1, 0, \dotsc){\cdot}(0,1,0,\dotsc,0,\dotsc) =
	$$
	$$
		= (x^n_0x_0, \dotsc, x^n_0x_n + \dotsc + x^n_nx_0, x^n_0x_{n+1} + \dotsc + x^n_{n}x_1 + x^n_{n+1}x_0, \dotsc) = (0, 0, \dotsc, 0, x^{n+1}_{n+1} = 1, 0, \dotsc )
	$$
	Таким образом, мы можем доказать последнее свойство: $\forall a = (a_0, a_1,\dotsc, a_n, \dotsc) \in K^{\infty}$:
	$$
		a = (a_0, 0, \dotsc, 0, \dotsc) + (0,a_1,0,\dotsc, 0, \dotsc) + \dotsc + (0,0,\dotsc, 0,a_n,0, \dotsc) + \dotsc
	$$ 
	$$
		\forall n \geq 0, \, (0, \dotsc, 0,\underset{n}{a_n},0, \dotsc ) = (\underset{0}{a_n}, 0, \dotsc, 0, \dotsc){\cdot}(0, \dotsc, 0,\underset{n}{1},0, \dotsc) = a_n{\cdot}x^n \Rightarrow
	$$
	$$
		\Rightarrow a = a_0 + a_1{\cdot}x + \dotsc + a_n{\cdot}x^n + \dotsc
	$$
	Коэффициенты в этой линейной комбинации однозначно определяются элементами исходной последовательности $a$, поскольку если мы проведем все выкладки в обратную сторону, то получим в точности такую же последовательность $a \Rightarrow$ всякая финитная последовательность, единственным способом представляется в виде линейной комбинации одночленов;
\end{enumerate}
Следовательно, $K^{\infty} = K[x]$.

\section*{Алгебраические свойства кольца многочленов}
\begin{defn}
	Кольцо $K$ называется \uwave{целостным} или \uwave{областью целостности}, если $K$ это коммутативное, ассоциативное кольцо с единицей без делителей нуля.
\end{defn}
\textbf{Примеры целостных колец}:
\begin{enumerate}[label=\arabic*)]
	\item $K = \MZ$;
	\item $K$ - любое поле, так как в поле любой элемент обратим $\Rightarrow$ не является делителем нуля (обратное не верно, смотри $K = \MZ$);
\end{enumerate}
Далее везде будем считать, что кольцо коэффициентов $K$ является областью целостности.

\begin{prop}
	Пусть $f,g \in K[x], \, f,g \neq 0$, тогда $f{\cdot}g \neq 0, \, \deg(f{\cdot}g) = \deg(f) + \deg(g)$.
\end{prop}
\begin{proof}
	Пусть верно:
	$$
		f = a_0 + a_1x + \dotsc + a_n x^n, \, a_n \neq 0, \, \deg(f) = n
	$$
	$$
		g = b_0 + b_1x + \dotsc  + b_m x^m, \, b_m \neq 0, \, \deg(g) = m
	$$
	Перемножим многочлены:
	$$
		f{\cdot}g = \ddsum{\substack{i = 0,\dotsc,n\\j = 0,\dotsc,m}}{}a_i{\cdot}b_j{\cdot}x^{i + j} = a_nb_mx^{n + m} + \text{члены меньших степеней}, \, a_nb_m \neq 0
	$$
	Старший член степени $n + m$ ни с кем не сокращается, поскольку остальные члены - меньших степеней. Таким образом, $f{\cdot}g \neq 0$ и $\deg(f{\cdot}g) = n + m = \deg(f) + \deg(g)$.
\end{proof}
\begin{rem}
	Мы предполагаем, что $f,g \neq 0$, но на самом деле формула верна и когда $f = 0$ или $g = 0$, если договориться, что $\deg(0) = -\infty$ и $\forall k > 0, \, -\infty + k = -\infty$.
\end{rem}

\begin{corollary}
	$K[x]$ над областью целостности $K$ - тоже является областью целостности.
\end{corollary}
\begin{proof}
	Следует сразу из предыдущего утверждения, потому что предложение говорит, что если есть два ненулевых многочлена, то их произведение тоже ненулевое $\Rightarrow$ нет делителей нуля.
\end{proof}
\begin{corollary}
	$K[x]^\times = K^\times$.
\end{corollary}
\begin{proof}\hfill\\
	$(\Leftarrow)$ Очевидно, поскольку $\forall f \in K^{\times}, \, \exists \, g \in K^{\times} \colon f{\cdot}g = 1 \Rightarrow f,g \in K[x], \, f{\cdot}g = 1 \Rightarrow f,g \in K[x]^{\times}$.
	
	$(\Rightarrow)$ Пусть $f{\cdot}g = 1$, тогда используя формулу из утверждения мы получаем:
	$$
		\deg(f) + \deg(g) = \deg(f{\cdot}g) = \deg(1) = 0
	$$	
	$$
		\deg(f), \deg(g) \geq 0 \Rightarrow \deg(f) = \deg(g) = 0 \Rightarrow f,g \in K
	$$
	То есть, такие многочлены лежат в $K$ и их произведение равно $1 \Rightarrow$ они обратимы в $K$.\\ Следовательно: $f$ обратим в $K[x] \Rightarrow f \in K$ и обратим в $K \Rightarrow K[x]^\times = K^\times$. 
\end{proof}
\begin{prop}
	Пусть $f,g \in K[x], \, f,g \neq 0$, тогда $\deg(f + g) \leq \max\{\deg(f),\deg(g)\}$.
\end{prop}
\begin{proof}
	Если $\deg(f) \neq \deg(g)$, то $\deg(f + g) = \max\{\deg(f),\deg(g)\}$, поскольку последним ненулевым элементом будет тот, у которого максимальный номер среди $f$ и $g$. Если же $\deg(f) = \deg(g)$, то возможна ситуация, когда это два многочлена вида:
	$$
		f(x) = a_nx^n + b_{n-1}x^{n-1}\dotsc + b_0, \, g(x) = -a_nx^n + c_{n-1}x^{n-1} + \dotsc + c_0 \Rightarrow \deg(f + g) < \deg(f) = \deg(g)
	$$
\end{proof}

Далее будем считать, что $K$ - это поле. Всякий многочлен $f = a_0 + a_1x + \dotsc + a_n x^n \in K[x]$ задает функцию $f \colon K \to K$ (обозначаемую той же буквой), которая определяется следующим образом:
$$
	\forall c \in K, \, f(c) = a_0 + a_1{\cdot}c + a_2{\cdot}c^2 + \dotsc + a_n{\cdot}c^n \in K
$$
\begin{defn}
	\uwave{Полиномиальными функциями} называются такие функции, которые задаются с помощью некоторого многочлена.
\end{defn}

\subsection*{Задача о полиномиальной интерполяции}
Пусть дано $n$ различных элементов $x_1, \dotsc, x_n \in K$ и ещё $n$ элементов (не обязательно различных) $y_1, \dotsc, y_n \in K$. Требуется найти (полиномиальную) функцию $f$, для которой $\forall i= \overline{1,n}, \, f(x_i) = y_i$.
\begin{theorem}(\textbf{Об интерполяции})
	Для любых различных элементов $x_1, \dotsc, x_n \in K$ и $\forall y_1, \dotsc, y_n \in K$ существует ровно один многочлен $f \in K[x], \, \deg(f) < n$, для которого выполнено: 
	$$
		\forall i= \overline{1,n}, \, f(x_i) = y_i
	$$
\end{theorem}
\begin{rem}
	Элементы $x_1,\dotsc,x_n$ поля $K$ принято называть \uwave{узлами интерполяции}.
\end{rem}
\begin{proof}
	Будем искать многочлен $f$ в виде линейной комбинации одночленов, степени меньше $n$:
	$$
		f = a_0 + a_1x + a_2 x^2 + \dotsc + a_{n-1}x^{n-1}
	$$
	\textbf{\uline{Условия интерполяции}}: 
	$$
		\begin{matrix}
			f(x_1) &=& a_0 &+& a_1x_1 &+& a_2 x_1^2 &+& \dotsc &+& a_{n-1}x_1^{n-1} &=& y_1\\
			f(x_2) &=& a_0 &+& a_1x_2 &+& a_2 x_2^2 &+& \dotsc &+& a_{n-1}x_2^{n-1} &=& y_2\\
			\vdots &\vdots& \vdots &\vdots&  \vdots &\vdots&  \vdots &\vdots&  \ddots &\vdots& \vdots &\vdots& \vdots\\
			f(x_n) &=& a_0 &+&  a_1x_n &+&  a_2 x_n^2 &+&  \dotsc &+&  a_{n-1}x_n^{n-1} &=& y_n
		\end{matrix}
	$$
	Можно посмотреть на эту систему, как на квадратную СЛУ относительно неизвестных $a_0, \dotsc, a_{n-1}$. Она квадратная, так как $n$ уравнений и $n$ неизвестных. Решим её методом Крамера и рассмотрим определитель матрицы коэффициентов:
	$$
		\Delta = 
		\begin{vmatrix}
			1 & x_1 & x_1^2 & \dotsc & x_1^{n-1}\\
			1 & x_2 & x_2^2 & \dotsc & x_2^{n-1}\\
			\vdots & \vdots & \vdots & \ddots & \vdots \\
			1 & x_n & x_n^2 & \dotsc & x_n^{n-1}
		\end{vmatrix} = V(x_1,x_2,\dotsc, x_n) = \prod\limits_{n \geq i > j \geq 1}(x_i - x_j) \neq 0
	$$
	Видим, что $\Delta$ с точностью до транспонирования это определитель Вандермонда, а по предположению, что все $x_i$ - различны получается, что этот определитель не равен нулю. Тогда по правилу Крамера, СЛУ - определена $\Rightarrow \exists!$ набор коэффициентов $a_0, a_1,\dotsc,a_{n-1}$, для которого соответствующий многочлен удовлетворяет условиям интерполяции.
\end{proof}

Найдем многочлен $f$ по формулам Крамера:
$$
	\forall j = \ovl{0,n-1}, \, a_j = \dfrac{\Delta_j}{\Delta}, \, \Delta = V(x_1,\dotsc, x_n) 
$$
$$
	\Delta_j = 
	\begin{vmatrix}
		1 & x_1 & x_1^2 & \dotsc & x_1^{j-1}& y_1 & x_1^{j+1} & \dotsc & x_1^{n-1}\\
		1 & x_2 & x_2^2 & \dotsc & x_2^{j-1}& y_2 & x_2^{j+1} & \dotsc &x_2^{n-1}\\
		\vdots & \vdots & \vdots & \ddots & \vdots 	& \vdots & \vdots & \ddots & \vdots \\
		1 & x_n & x_n^2 & \dotsc & x_n^{j-1}& y_n & x_n^{j+1} & \dotsc &x_n^{n-1}
	\end{vmatrix}
$$
где $j+1$-ый столбец заменен на столбец свободных членов $y$, поскольку нумерация столбцов сдвинута на $1$. Посчитаем этот определитель в явном виде, разложив по $j+1$-ому столбцу:
$$
	\Delta_j = \ddsum{i = 1}{n}y_i{\cdot}(-1)^{i + j + 1}{\cdot}
	\begin{vmatrix}
		1 & x_1 & x_1^2 & \dotsc & x_1^{j-1}&  x_1^{j+1} & \dotsc & x_1^{n-1}\\
		\vdots & \vdots & \vdots & \ddots &  \vdots & \vdots & \ddots & \vdots \\
		1 & x_{i-1} & x_{i-1}^2 & \dotsc & x_{i-1}^{j-1}&  x_{i-1}^{j+1} & \dotsc &x_{i-1}^{n-1}\\
		1 & x_{i+1} & x_{i+1}^2 & \dotsc & x_{i+1}^{j-1}&  x_{i+1}^{j+1} & \dotsc &x_{i+1}^{n-1}\\
		\vdots & \vdots & \vdots & \ddots &  \vdots & \vdots & \ddots & \vdots \\
		1 & x_n & x_n^2 & \dotsc & x_n^{j-1} & x_n^{j+1} & \dotsc &x_n^{n-1}
	\end{vmatrix} = \ddsum{i = 1}{n}y_i{\cdot}(-1)^{i + j + 1}{\cdot}\Delta_{-ij} \Rightarrow
$$
$$
	\Rightarrow f(x) = \ddsum{j = 0 }{n-1}a_j{\cdot}x^j = \dfrac{1}{\Delta}\ddsum{\substack{j = 0,\dotsc,n-1\\i = 0,\dotsc,n}}{}(-1)^{i + j + 1}{\cdot}y_i{\cdot}\Delta_{-ij} {\cdot}x^j =  \dfrac{1}{\Delta}\ddsum{i =1}{n}y_i{\cdot}\ddsum{j = 0}{n-1}(-1)^{i + j + 1}{\cdot}x^j{\cdot}\Delta_{-ij} 
$$
Здесь мы получаем формулу разложения определителя по строке $i$, состоящей из членов $x^j$, тогда:
$$
	f(x) = \dfrac{1}{\Delta}\ddsum{i =1}{n}y_i{\cdot}
	\begin{vmatrix}
		1 & x_1 & x_1^2 & \dotsc & x_1^j & \dotsc & x_1^{n-1}\\
		\vdots & \vdots & \vdots & \ddots  & \vdots & \ddots & \vdots \\
		1 & x 	& x^2 	& \dotsc & x^j & \dotsc &x^{n-1}\\
		\vdots & \vdots & \vdots & \ddots  & \vdots & \ddots & \vdots \\
		1 & x_n & x_n^2 & \dotsc & x_n^j & \dotsc &x_n^{n-1}
	\end{vmatrix} = \dfrac{1}{\Delta}\ddsum{i =1}{n}y_i{\cdot}V(x_1,\dotsc,\underset{i}{x}, \dotsc, x_n) =
$$
$$
	=	\ddsum{i =1}{n}y_i{\cdot} \dfrac{V(x_1, \dotsc, x, \dotsc, x_n)}{V(x_1,\dotsc,x_i, \dotsc, x_n)} = 	\ddsum{i =1}{n}y_i{\cdot} \dfrac{\prod\limits_{k > l,\, k,l \neq i}(x_k - x_l){\cdot}\prod\limits_{k > i}(x_k - x)\prod\limits_{i > l}(x - x_l)}{\prod\limits_{k > l,\, k,l \neq i}(x_k - x_l){\cdot}\prod\limits_{k > i}(x_k - x_i)\prod\limits_{i > l}(x_i - x_l)} = 
$$
$$
	= \ddsum{i =1}{n}y_i{\cdot} \dfrac{(x - x_1){\cdot}\dotsc{\cdot}(x - x_{i-1}){\cdot}(x - x_{i +1}){\cdot}\dotsc{\cdot}(x - x_n)}{(x_i - x_1){\cdot}\dotsc{\cdot}(x_i - x_{i-1}){\cdot}(x_i - x_{i +1}){\cdot}\dotsc{\cdot}(x_i - x_n)}
$$
где в числителе мы поменяли порядок $x$ и $x_k$ из-за таких же перестановок в знаменателе.
\begin{defn}
	\uwave{Интерполяционной формулой Лагранжа} называется явный вид многочлена $f$ построенного по формулам Крамера для задачи полиномиальной интерполяции:
	$$
		f(x) = \ddsum{i =1}{n}y_i{\cdot} \dfrac{(x - x_1){\cdot}\dotsc{\cdot}(x - x_{i-1}){\cdot}(x - x_{i +1}){\cdot}\dotsc{\cdot}(x - x_n)}{(x_i - x_1){\cdot}\dotsc{\cdot}(x_i - x_{i-1}){\cdot}(x_i - x_{i +1}){\cdot}\dotsc{\cdot}(x_i - x_n)}
	$$
\end{defn}
\end{document}